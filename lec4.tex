\begin{lemma} [О несамодвойственной функции] Пусть $f(\XXX)\not\in S$, тогда $0,\,1 \in [\{f, \bar{x}\}]$
\end{lemma}
\begin{proof} 
    Пусть $f(\XXX)\not\in S$, тогда  $f^*(\XXX) = \overline{f(\bar{x}_1,\ldots,\bar{x}_n)} \neq f(\XXX) \Rightarrow \exists \sigma = (\sigma_1,\ldots,\sigma_n),$ т.ч. $\overline{f(\bar{\sigma}_1,\ldots,\bar{\sigma}_n)} \neq f(\sigma_1,\ldots,\sigma_n) \Rightarrow f(\bar{\sigma}_1,\ldots,\bar{\sigma}_n) = f(\sigma_1,\ldots,\sigma_n) = C.$\\
    Будем считать, что $(\SIG) = (\underbrace{0,\ldots,0}_k,\underbrace{1,\ldots,1}_{n-k})$.\\
    Пусть $g(x) = f(\underbrace{x,\ldots,x}_k,\underbrace{ \bar{x},\ldots,\bar{x}}_{n-k}).$\\
    $g(0) = f(\underbrace{0,\ldots,0}_k,\underbrace{ 1,\ldots,1}_{n-k}) = f(\SIG),\\
    g(1)= f(\underbrace{1,\ldots,1}_k,\underbrace{ 0,\ldots,0}_{n-k}) = f(\bar{\sigma}_1,\ldots,\bar{\sigma}_n)$,\\ значит, 
    $g(0) = g(1) = C, $ причём $g$ задаётся формулой над $\{f,\,\bar{x}\} \Rightarrow g \in S$.\\
    Получаем $C \in  [\{f, \bar{x}\}] \Rightarrow \bar{C} \in [\{f, \bar{x}\}] \Rightarrow 0,\,1 \in [\{f, \bar{x}\}].$
\end{proof}

\subsection{Линейные функции.}
\begin{definition}
Булева функция называется линейной, если степень её полинома Жегалкина не превосходит $1$.
\end{definition}
Здесь под степень полинома Жегалкина понимается максимальная длина слагаемого в нём или, говоря алгебраическим языком, его степень как многочлена над $\mathbb{Z}_2$. Например, степень полинома $xyz \oplus x \oplus 1$ равна $3$. 
\begin{definition}
$L$ --- класс всех линейных булевых функций.
\end{definition}
\begin{proposition}
1) $0,\, 1,\, x,\, \bar{x},\, x\oplus y,\, x \sim y \in L$, 2) $x \to y,\, x \vee y,\, x\& y \not\in L.$
\end{proposition}
\begin{proof}
Первая часть утверждения очевидна, кроме утверждения про функцию $x \sim y$. Чтобы доказать оставшееся, представим следующие функции в виде полиномов:\\
$x \sim y = x \oplus y \oplus 1 ~ (\deg = 1),\\
x \to y = \bar{x} \vee xy = xy \oplus x \oplus 1 ~ (\deg = 2),\\
x \vee y = xy \oplus x \oplus y ~ (\deg = 2)$.
%дописать про последнюю функцию x\& y !!!
\end{proof}