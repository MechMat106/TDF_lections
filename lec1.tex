\section{Лекция 1.}
\subsection{Определение булевой функции.}
$E=\lbrace0,1\rbrace$ \\
\begin{definition}
$f(\XXX)\in{E}$ --- функция алгебры логики {\bf (булева функция)} 

$x_{i}\in{E} ~ \forall i=1,\ldots,n$\\
\end{definition}

\begin{definition}
$P_{2}$ --- множество всех булевых функций.\\
\end{definition}
\begin{definition}
$E^{n}=\{(\SIG)|~ \sigma_{i}\in E; ~ i=1,\ldots,n\}$ \\
\end{definition}
Булева функция задает отображение $f\colon E^{n}\rightarrow{E}$. Это можно проиллюстрировать таблицей возможных значений $f$ на различных наборах переменных:\\

$$\begin{array}{|ccccc|c|}
\hline
x_1 & x_2 & \ldots & x_{n-1} & x_n & f(\XXX)  \\
0 & 0 &\ldots & 0 & 1 & 0 ~ \textbf{или} ~ 1 \\
0 & 0 &\ldots & 1 & 1 & 0 ~ \textbf{или} ~ 1 \\
\ldots & \ldots & \ldots & \ldots & \ldots & \ldots\\
1 & 1 &\ldots & 1 & 1 & 0 ~ \textbf{или} ~ 1 \\
\hline
\end{array}$$
\\
\\
\begin{statement}
 $|P_2(\XXX)|=2^{2^{n}}$. \\
\end{statement}
\begin{proof}
Очевидно.
\end{proof}
\subsection{Существенные и фиктивные переменные.} 
\begin{definition}
	Пусть $f(\XXX)$ -- булева функция. 
	Тогда $x_i$ называется \textbf{существенной} переменной для $f$, если:  $\exists{\sigma_1,\sigma_2, \ldots \sigma_{i-1}, \sigma_{i+1}, \ldots, \sigma_{n}}\in\{0,1\}$, такие, что: 

$ f(\sigma_1,\sigma_2, \ldots ,\sigma_{i-1}, 0, \sigma_{i+1}, \ldots, \sigma_{n})\neq f(\sigma_1,\sigma_2, \ldots, \sigma_{i-1}, 1, \sigma_{i+1}, \ldots, \sigma_{n}).$
	В противном случае переменная называется \textbf{фиктивной} (пример придумать не очень сложно).
\end{definition}
\begin{enumerate}
	\item 
	Пусть $x_i$ --- фиктивная переменная для $f$. 

    Рассмотрим функцию $g$:
    $g(x_1,x_2,\ldots,x_{i-1},x_{i+1}, \ldots, x_{n}): \\
    g(\sigma_1,\sigma_2 \ldots \sigma_{i-1}, \sigma_{i+1}, \ldots, \sigma_{n})=f(\sigma_1,\sigma_2 \ldots \sigma_{i-1},0, \sigma_{i+1}, \ldots, \sigma_{n}) =f(\sigma_1,\sigma_2 \ldots \sigma_{i-1},1, \sigma_{i+1}, \ldots, \sigma_{n}) $
    Тогда говорят, что  \textbf{$g$ получена из $f$ удалением фиктивной переменной $x_i$}. \\

    \item 
     Пусть $f(\XXX)$ -- булева функция. Также, пусть имеется $y \neq \XXX $. Рассмотрим функцию $h(\XXX,y)$: \\
$h(\SIG,\sigma)=f(\SIG)$ \\
Тогда говорим, что \textbf{$h$ получена из $f$ добавлением фиктивной переменной  $y$.}

\end{enumerate}
\begin{definition}
	Две булевы функции называются \textbf{равными}, если они могут быть получены друг из друга с помощью некоторого числа операций добавления или удаления фиктивных переменных. \\
\end{definition}
\subsection{Элементарные функции:} 
\begin{enumerate}
	\item От одной переменной.
	$$
    \begin{array}{|c|c|c|c|c|}
    \hline
    x & 0 & x & \bar{x} & 1 \\
    \hline
    0 & 0 & 0 & 1 & 1 \\
    \hline
    1 & 0 & 1 & 0 & 1 \\
    \hline
    \end{array}
    $$
    \item От двух переменных:
    $$
    \begin{array}{|c|c|c|c|c|c|c|c|c|c|}
	\hline
	 x & y & xy & x\vee y & x\oplus y & x\sim y & x\rightarrow y & x|y & x\downarrow y\\
	\hline
	 0 & 0 & 0 & 0 & 0 & 1 &  1 & 1 & 1 \\
	\hline
	 0 & 1 & 0 & 1 & 1 & 0 &  1 & 1 & 0 \\
	\hline
	 1 & 0 & 0 & 1 & 1 & 0 &  0 & 1 & 0  \\
	\hline
	 1 & 1 & 1 & 1 & 0 & 1 &  1 & 0 & 0 \\
	\hline
	\end{array}
	$$
	\item
	От трех переменных(функция "медиана"):
	$$
	\begin{array}{rrr|c}
	x~~ & y~~ & z~~ & f(x,y,z)\\
	\hline
	\begin{array}{r} % Вложенная таблица для каждого столбца
	0\\ 0\\ 0\\ 0\\ 1\\ 1\\ 1\\ 1\\
	\end{array}
	&
	\begin{array}{r}
	0\\ 0\\ 1\\ 1\\ 0\\ 0\\ 1\\ 1\\
	\end{array}
	&
	\begin{array}{r}
	0\\ 1\\ 0\\ 1\\ 0\\ 1\\ 0\\ 1\\
	\end{array}
	&
	\begin{array}{r}
	0\\ 0\\ 0\\ 1\\ 0\\ 1\\ 1\\ 1\\
	\end{array}
	\end{array}
	$$
\end{enumerate}	


\subsection{Формула над системой булевых функций.}
$F=\{f_{1}(x_{1},x_{2},...,x_{n_1});f_{2}(x_{1},x_{2},...,x_{n_2});...;f_{n}(x_{1},x_{2},...,x_{n_n})\}\subseteq P_2$ -- некоторое множество булевых функций, таких что каждой булевой функции $f_{i}(x_{1},x_{2},...,x_{n_i})$ сопоставляем функциональный символ $f_{i}$.
\begin{definition}

	\textit{Формулой над F} называется строка символов, состоящая из любых символов-переменных, обозначающих $f_1,...,f_n$ и вспомогательных символов $"("$,$")"$ ,$","$, определяемое индуктивным образом: 

\textbf{База индукции:} символ любой переменной -- правильная формула над F.

\textbf{Индуктивное предположение: } пусть $F_1,F_2,...,F_{n_i}$ -- некоторые формулы над F, тогда $f_i(F_1,F_2,...,F_{n_i})$ -- тоже формула над F.

\end{definition}
\begin{example}
	$((\overline{x\vee y}) \& (z\longrightarrow y ))$ --- формула над $\{x \vee y; x \& y, x \longrightarrow y, \overline{x} \}$\\
\end{example}
Конъюнкция имеет приоритет над дизъюнкцией.\\
\newpage
Значения формулы на наборе значений переменных, входящих в формулу, определяется индуктивным образом.

\textbf{База индукции:} если $f$ --- тривиальная, то все очевидно.

\textbf{Индуктивное предположение:} пусть $F_1,F_2, \ldots, F_n$ -- формулы, для которых данное понятие уже определено. \\
$F=f_i(F_1,F_2, \ldots, {F_{n_i}});$ \\
$\XXX$ -- все переменные, содержащиеся в F. \\
$\Omega=(\SIG)$ -- набор значений $\XXX$. \\
$\Omega_j$ -- поднабор значений из $\Omega$ для переменных, содержащихся в формуле $F_j$. \\
$b_j$ -- значение функции $F_j$ на наборе $\Omega_j$. \\
Тогда значение $F$ на наборе $\Omega$ равно $f_i(b_1, \ldots, {b_n}_i)$ \\
Пусть $F$ -- формула над $\Phi$, содержащая символы переменных $x_1, \ldots, x_n$. Тогда $F$ реализует функцию $f(\XXX)$, т.ч для любого набора $(\sigma_1, \ldots, \sigma_n)$ значений $x_1,...,x_n$ значение $f(\sigma_1, \ldots, \sigma_n)$ равно значению формулы $F$ на $\sigma_1, \ldots, \sigma_n$. \\
$f$ получается из $\Phi$ с помощью операции суперпозиции, если F реализуется некоторой нетривиальной формулой над $\Phi$.

\begin{definition}
Две формулы $F_1$ и $F_2$ называются {\bf эквивалентными}, если они реализуют одинаковые функции.
\end{definition}


$\ast \in \{\vee, \&, \oplus, \sim \}$

\begin{enumerate}
	\item 
		$x \ast y = y \ast x$ (коммутативность)
	\item 
		$x \ast (y \ast z) = (x \ast y) \ast z$ (ассоциативность)
	\item 
		$x (y \vee z) = xy \vee xz$

		$x (y \oplus z) = xy \oplus xz$

		$x \vee (y  \&  z) = (x \vee y) \& (x \vee z)$

		$x \vee (y  \sim  z) = (x \vee y) \sim (x \vee z)$ (дистрибутивность)
	\item
		$x \vee xy = x$ (поглощение)
	\item 
		$\overline{\overline{x}} = x$ (двойное отрицание)
	\item 
		$\overline{x \vee y} = \overline{x} \& \overline{y}$

		$\overline{x \& y} = \overline{x} \vee \overline{y}$ (закон де Моргана)
	\item 
		$x\overline{x} = 0$, \smallskip $x \vee \overline{x} = 1$, \smallskip $x \oplus \overline{x} = 1$, \smallskip $x \sim \overline{x} = 0$

		$xx = x$, \smallskip $x \vee x = x$, \smallskip $x \oplus x = 0$, \smallskip $x \sim x = 1$

		$x \& 1 = x$, \smallskip $x \vee 1 = 1$, \smallskip $x \oplus 1 = \overline{x}$, \smallskip $x \sim 1 = x$

		$x \& 0 = 0$, \smallskip $x \vee 0 = x$, \smallskip $x \oplus 0 = x$, \smallskip $x \sim 0 = \overline{x}$
\end{enumerate}