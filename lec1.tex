\section{Булевы функции}


\subsection{Определение булевой функции.}
Обозначим за $E$ множество $\lbrace 0, 1 \rbrace$.

\begin{df}
	$f(\XXX) \in {E}$ --- функция алгебры логики {\bf (булева функция)},
	где $x_{i} \in {E} ~ \forall i = 1, \ldots, n$ --- это отображение $f \colon E^{n} \rightarrow {E}$.
	Его можно проиллюстрировать таблицей возможных значений $f$ на различных наборах переменных:\\

	$$
	\begin{array}{|ccccc|c|}
		\hline
		x_1 & x_2 & \ldots & x_{n-1} & x_n & f(\XXX)  \\
		0 & 0 &\ldots & 0 & 1 & 0 ~ \textbf{или} ~ 1 \\
		0 & 0 &\ldots & 1 & 1 & 0 ~ \textbf{или} ~ 1 \\
		\ldots & \ldots & \ldots & \ldots & \ldots & \ldots\\
		1 & 1 &\ldots & 1 & 1 & 0 ~ \textbf{или} ~ 1 \\
		\hline
	\end{array}
	$$
\\
\\\\
\end{df}

\begin{df}
	$P_{2}$ --- множество всех булевых функций от произвольного конечного множества переменных.
	$P_2(n)$ --- множество всех булевых функций от $n$ переменных. 
\end{df}

\begin{df}
	$E^{n} = \{ (\SIG) |~ \sigma_{i} \in E; ~ i = 1, \ldots, n \}$ \\
\end{df}
 
\begin{stm}
	 $|P_2(n)| = 2^{2^{n}}$. \\
\end{stm}

\begin{proof}
	Очевидно.
\end{proof}


\subsection{Существенные и фиктивные переменные.} 
\begin{df}
	Пусть $f(\XXX)$ --- булева функция. 
	Тогда $x_i$ называется \textbf{существенной} переменной для $f$,
	если $\exists {\sigma_1, \sigma_2, \ldots \sigma_{i-1}, \sigma_{i+1}, \ldots, \sigma_{n}} \in \{ 0,1 \}:$
	$$
		f( \sigma_1, \sigma_2, \ldots, \sigma_{i-1}, 0, \sigma_{i+1}, \ldots, \sigma_{n})
		\neq f(\sigma_1, \sigma_2, \ldots, \sigma_{i-1}, 1, \sigma_{i+1}, \ldots, \sigma_{n}).
	$$
	В противном случае переменная называется \textbf{фиктивной}
	(пример придумать не очень сложно).
\end{df}

\begin{df}
	Пусть $x_i$ --- фиктивная переменная для $f$.
	Рассмотрим функцию
	\begin{gather*}
		\begin{split}
			g(x_1, x_2, \ldots, x_{i-1}, x_{i+1}, \ldots, x_{n})\colon &
			g(\sigma_1, \sigma_2, \ldots, \sigma_{i-1}, \sigma_{i+1}, \ldots, \sigma_{n}) = \\
			= f(\sigma_1, \sigma_2, \ldots, \sigma_{i-1}, 0, \sigma_{i+1}, \ldots, \sigma_{n})
			&= f(\sigma_1, \sigma_2, \ldots, \sigma_{i-1}, 1, \sigma_{i+1}, \ldots, \sigma_{n}).
		\end{split}
	\end{gather*}
	Тогда говорят, что \textbf{$g$ получена из $f$ удалением фиктивной переменной $x_i$}.
\end{df}

\begin{df}
	Пусть $f(\XXX)$ --- булева функция.
	Также, пусть имеется $y \neq \XXX $.
	Рассмотрим функцию $h(\XXX, y)$, $h(\SIG, \sigma) = f(\SIG)$.
	Тогда говорим, что \textbf{$h$ получена из $f$ добавлением фиктивной переменной  $y$.}
\end{df}

\begin{df}
	Две булевы функции называются \textbf{равными},
	если они могут быть получены друг из друга с помощью
	некоторого числа операций добавления или удаления фиктивных переменных. \\
\end{df}


\subsection{Элементарные функции} 
\begin{enumerate}
	\item От одной переменной:
	$$
	\begin{array}{|c|c|c|c|c|}
		\hline
		x & 0 & x & \bar{x} & 1 \\
		\hline
		0 & 0 & 0 & 1 & 1 \\
		\hline
		1 & 0 & 1 & 0 & 1 \\
		\hline
	\end{array}
	$$

	\item От двух переменных:
	$$
	\begin{array}{|c|c|c|c|c|c|c|c|c|c|}
		\hline
		 x & y & xy & x\vee y & x\oplus y & x\sim y & x\rightarrow y & x|y & x\downarrow y\\
		\hline
		 0 & 0 & 0 & 0 & 0 & 1 &  1 & 1 & 1 \\
		\hline
		 0 & 1 & 0 & 1 & 1 & 0 &  1 & 1 & 0 \\
		\hline
		 1 & 0 & 0 & 1 & 1 & 0 &  0 & 1 & 0  \\
		\hline
		 1 & 1 & 1 & 1 & 0 & 1 &  1 & 0 & 0 \\
		\hline
	\end{array}
	$$

	\item
	От трёх переменных (функция "медиана"):
	$$
	\begin{array}{rrr|c}
		x~~ & y~~ & z~~ & f(x,y,z)\\
		\hline
		\begin{array}{r} % Вложенная таблица для каждого столбца
			0\\ 0\\ 0\\ 0\\ 1\\ 1\\ 1\\ 1\\
		\end{array}
		&
		\begin{array}{r}
			0\\ 0\\ 1\\ 1\\ 0\\ 0\\ 1\\ 1\\
		\end{array}
		&
		\begin{array}{r}
			0\\ 1\\ 0\\ 1\\ 0\\ 1\\ 0\\ 1\\
		\end{array}
		&
		\begin{array}{r}
			0\\ 0\\ 0\\ 1\\ 0\\ 1\\ 1\\ 1\\
		\end{array}
	\end{array}
	$$
\end{enumerate}	


\subsection{Формула над системой булевых функций.}
$
	\Phi =
	\{
		f_{1}(x_{1}, x_{2}, \ldots, x_{n_1});
		f_{2}(x_{1}, x_{2}, \ldots, x_{n_2});
		\ldots;
		f_{n}(x_{1}, x_{2}, \ldots, x_{n_n})
	\} \subseteq P_2
$ --- некоторое множество булевых функций,
	таких что каждой булевой функции $f_{i}(x_{1}, x_{2}, \ldots, x_{n_i})$
	сопоставляем функциональный символ $f_{i}$.

\begin{df}
	\textbf{Формулой над $\Phi$} называется строка символов,
	состоящая из любых символов-переменных,
	обозначающих $f_1, \ldots, f_n$ и вспомогательных символов <<$($>>, <<$)$>>, <<$,$>>,
	определяемое индуктивным образом: 

	\textit{База индукции:} символ любой переменной --- правильная формула над $\Phi$.

	\textit{Индуктивное предположение:} пусть $F_1, F_2, \ldots, F_{n_i}$ --- некоторые формулы над $\Phi$,
	тогда $f_i(F_1, F_2, \ldots, F_{n_i})$ --- тоже формула над $\Phi$.
\end{df}

\begin{ex}
	$((\overline{x \vee y}) \& (z \rightarrow y))$ --- формула над $\{ x \vee y; x \& y, x \rightarrow y, \overline{x} \}$
\end{ex}
Конъюнкция имеет приоритет над дизъюнкцией. \\

%\newpage 
%зачем тут newpage? Не зачем, наверное.
%TODO:понять зачем newpage.

\begin{df}
	Значения формулы на наборе значений переменных,
	входящих в формулу, определяется индуктивным образом.
	
	\textit{База индукции:} если $f$ --- тривиальная, то все очевидно.

	\textit{Индуктивное предположение:} пусть $F_1,F_2, \ldots, F_n$ --- формулы,
	для которых данное понятие уже определено.
	\begin{align*}
		F &= f_i(F_1, F_2, \ldots, {F_{n_i}});\\
		\XXX &\,\text{--- все переменные, содержащиеся в $F$};\\
		\Omega &= (\SIG) \text{--- набор значений $\XXX$}; \\
		\Omega_j &\,\text{--- поднабор значений из $\Omega$ для переменных, содержащихся в формуле $F_j$;} \\
		b_j &\,\text{--- значение функции $F_j$ на наборе $\Omega_j$}. 
	\end{align*}

	Тогда значение $F$ на наборе $\Omega$ равно $f_i(b_1, \ldots, {b_n}_i)$.

	Пусть $F$ --- формула над $\Phi$, содержащая символы переменных $x_1, \ldots, x_n$.
	Тогда $F$ реализует функцию $f(\XXX)$, \sth для любого набора
	$(\SIG)$ значений $\XXX$ значение $f(\SIG)$ равно значению формулы $F$ на $\SIG$.
	$f$ получается из $\Phi$ с помощью операции суперпозиции,
	если $F$ реализуется некоторой нетривиальной формулой над $\Phi$.
\end{df}

\begin{df}
	Две формулы $F_1$ и $F_2$ называются {\bf эквивалентными},
	если они реализуют одинаковые функции.
\end{df}

Пусть $\ast \in \{\vee, \&, \oplus, \sim \}$ --- некоторая операция. Тогда $\ast$ имеет следующие

\begin{props}
	\prpy{коммутативность} 
		$x \ast y = y \ast x$
	\prpy{ассоциативность}
		$x \ast (y \ast z) = (x \ast y) \ast z$
	\prpy{дистрибутивность}
		\begin{align*}
			x (y \vee z) &= xy \vee xz \\
			x (y \oplus z) &= xy \oplus xz \\
			x \vee (y  \&  z) &= (x \vee y) \& (x \vee z) \\
			x \vee (y  \sim  z) &= (x \vee y) \sim (x \vee z) \\
		\end{align*}
	\prpy{поглощение}
		$x \vee xy = x$
	\prpy{двойное отрицание} 
		$\overline{\overline{x}} = x$
	\prpy{закон де Моргана}
		\begin{align*}
			\overline{x \vee y} &= \overline{x} \& \overline{y} \\
			\overline{x \& y} &= \overline{x} \vee \overline{y} \\ 
		\end{align*}
	\prpy{}
		\begin{align*} 
			x\overline{x} &= 0, & x \vee \overline{x} &= 1, & x \oplus \overline{x} &= 1, & x \sim \overline{x} &= 0 \\
			xx &= x, & x \vee x &= x, & x \oplus x &= 0, & x \sim x &= 1 \\
			x \& 1 &= x, & x \vee 1 &= 1, & x \oplus 1 &= \overline{x}, & x \sim 1 &= x \\
			x \& 0 &= 0, & x \vee 0 &= x, & x \oplus 0 &= x, & x \sim 0 &= \overline{x}
		\end{align*}
\end{props}
