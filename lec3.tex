%\section{Лекция 3.}
\subsection{Полином Жегалкина.} 
Пусть $f(x_1, \ldots, x_n)$ --- булева функция. %$f\neq 0$.
\begin{definition} \textit{Полиномом Жегалкина} функции $f$ называется полином $P$ с коэффициентами в $\{0,1\}$ от переменных $\XXX$ степени не выше $n$, такой что $f(\XXX) = P(\XXX)$. \end{definition}
\begin{statement}
    Полином Жегалкина существует для любой функции $f(\XXX)$.
\end{statement}
\begin{proof}
$f(x_1, \ldots, x_n)=\bigvee \limits_{f(\sigma_1 \ldots \sigma_n)=1} x^{\sigma_1} \ldots x^{\sigma_n}=\bigoplus \limits_{f(\sigma_1, \ldots, \sigma_n)=1} x^{\sigma_1} \ldots x^{\sigma_n}=$ 
\begin{flushright}
$
=\bigoplus \limits_{f(\sigma_1, \ldots, \sigma_n)=1} (x_1\oplus \bar{\sigma_1})\ldots(x_n\oplus \bar{\sigma_n})=\bigoplus \limits^n_{k=0} \big( \bigoplus \limits_{i_1<i_2< \ldots<i_k} c_{i_1, \ldots, i_k}x_{i_1} \ldots x_{i_k}\big) $, где  $c_{i_1, \ldots, i_k} \in \{0,1\}$.
\end{flushright} 
В этой сумме слагаемое с $k=0$ соответствует произведению пустого множества переменных, то есть свободному члену.
\end{proof}

Из определения следует, что если $f$ --- константа, то её полином Жегалкина имеет степень $0$, то есть равен $1$ или $0$ (в зависимости от того, какой константой является $f$, разумеется). 
\begin{statement}
	Для каждой булевой функции от $n$ переменных существует единственный полином Жегалкина.
\end{statement}
\begin{proof}
    Как было доказано выше, для каждой функции полином Жегалкина существует. Далее, очевидно, что разным функциям соответствуют разные полиномы Жегалкина. Покажем, что всевозможных полиномов степени не выше $n$ от переменных $\XXX$ ровно столько же, сколько всевозможных булевых функций от этих переменных. \\
    1) $|P_2(n)|=2^{2^{n}}$.\\
    2) Каждый коэффициент $c_{i_1, \ldots, i_k}$ соответствует подмножеству $\{x_{i_1}, \ldots, x_{i_k}\}$ (возможно пустому) из множества переменных $\{\XXX\}$. Таких подмножеств $2^{n}$. Каждый коэффициент принимает значения $0$ или $1$, значит, всего полиномов $2^{2^n}$. Отсюда всё очевидно.
%	Полином Жегалкина единственен, так как $c_{i_1}, \ldots, c_{i_n}$ однозначно определены $2^n$ -  коэффицентов $2^{2^n}$ - набор значений коэффицентов $2^{2^n}$ - количество функций от $n$ - переменных $\Rightarrow$ для каждой функции $\exists !$ полином Жегалкина. 
\end{proof}
\section{Замкнутые классы булевых функций}
\subsection{Функции, сохраняющие ноль и единицу}
\begin{definition}
	$f$ сохраняет 0, если $f(0,\ldots, 0)=0$.
	$T_0$ --- множество всех функций, сохраняющих ноль. Например, $0,\,x,\,x \& y,\, x \vee y,\, x \oplus y$.
\end{definition}
\begin{definition}
	Селекторная функция --- функция, тождественна равная переменной. 
\end{definition}
\begin{lemma}
	$T_0 $ --- замкнуто. 
\end{lemma}
\begin{proof}
	Тождественная функция содержится в $T_0$. Значит, надо проверить, что если $f(x_1,\ldots, x_n),g_1, \ldots, g_n \in T_0, \text{, то~} f(g_1, \ldots, g_n) \in T_0.$ \\
	Можем полагать, что $g_1, \ldots, g_n$ зависят от одних и тех же переменных: $x_1, \ldots, x_n$ (иначе можно добавить переменные в качестве фиктивных). Тогда:\\
	$f(g_1(x_1, \ldots, x_n),\ldots, g_n(x_1, \ldots, x_n))=h(x_1, \ldots, x_n) \\
	h(0, \ldots, 0)=f(g_1(0,\ldots,0),\ldots,g_n(0, \ldots, 0))=f(0,\ldots,0)=0. \Rightarrow h \in T_0. 
$
\end{proof}
\begin{definition}
	$f$ сохраняет 1, если $f(1,\ldots, 1)=1$.
	Обозначим за $T_1$ множество всех функций, сохраняющих единицу. Например, $1,\,x,\,xy\rightarrow y, \,x \vee y$.
\end{definition}
\begin{lemma}
	$T_1 $ --- замкнуто. 
\end{lemma}
\begin{proof}
	Аналогично предыдущей лемме. 
\end{proof}
\subsection{Монотонные функции.}	
Определим правило сравнения на наборах из нулей и единиц.\\
$\sigma'=\{\sigma_1',\ldots,\sigma_n'\}, \,\sigma''= \{\sigma_1'',\ldots,\sigma_n''\} \in \{0,1\}^n.$ \\
Будем говорить, что $\sigma'\leq\sigma'',\, \text{если} ~ \forall i \in \{1,\ldots,n\} ~ \sigma_i ' \leq\sigma_i '' .$ \\
Заметим, что существуют несравнимые наборы, например: $(101)$ и $(010)$. 
\begin{definition}
$f$ --- монотонная, если для любых $\sigma' \text{ и } \sigma'' ~ \text{таких, что} ~ \sigma'\leq \sigma'' $ выполняется, что $f(\sigma')\leqslant f(\sigma'').$
\end{definition}
\begin{lemma}
	$M$ является замкнутым классом. 
\end{lemma}
\begin{proof}
	%Воткнуть в обозначения лекций я так и не смог, потому частично взял с Лупанова. 
Тождественная функция содержится в $M$. Значит, осталось проверить, что если
$f(x_1, \ldots,x_n), \,g_1, \ldots, g_n \in M$, то $h=f(g_1,\ldots,g_n) \in M $. Можно считать, что  $g_1,\ldots g_n$  --- функции от одного и того же количества переменных, в противном случае недостающие переменные можно добавить в качестве несущественных.  Выберем произвольные различные наборы $ 
\sigma'=\{\sigma_1',\ldots,\sigma_n'\}, \,\sigma''= \{\sigma_1'',\ldots,\sigma_n''\}$, такие что $ \sigma' \leq \sigma''.\\
$Рассмотрим $ h(\sigma')=f(g_1(\sigma'),g_2(\sigma'), \ldots, g_n\sigma')) ~\text{и}~ h(\sigma'')=f(g_1(\sigma''),\ldots,g_n(\sigma'')). \\
g_i(\sigma')<g_i(\sigma''), $ так как $ g_i$ --- монотонная. $ f(g_1(\sigma'),g_2(\sigma'), \ldots, g_n(\sigma')) \leq f(g_1(\sigma''),g_2(\sigma''), \ldots, g_n(\sigma''))$, так как f --- монотонная, то $h$ --- тоже монотонная. 
\end{proof}
\begin{lemma}[О немонотонных функциях]
	$f(x_1,\ldots, x_n)\notin M. $ Тогда $\bar{x} \in [\{f;0;1\}].$
\end{lemma}	
\begin{proof}
	$f \notin M \Rightarrow \exists \,\sigma',\sigma'': \sigma' \leq \sigma'',\,$ $f(\sigma')=1,\, f(\sigma'')=0.$
%,\, \text{где }  \sigma',  \sigma'' \text{ ---разные}. $\\ кажется, это лишнее?
Без ограничения общности будем считать, что $\sigma' ~\text{и} ~ \sigma''$ устроены следующим образом: \\$ 
	\sigma'=(0,\ldots,0,\ldots,0,1,\ldots,1) \\
	\sigma''=(\underbrace{1,\ldots,1}_k,\underbrace{0,\ldots,0}_s,\underbrace{1,\ldots,1}_{n-k-s}) \\
	g(x)=f(\underbrace{x, \ldots, x}_k, \underbrace{0, \ldots, 0}_s,\underbrace{1,\ldots,1}_{n-k-s}) =\bar{x},$ так как $g(0)=1 ~ \text{и} ~ g(1)=0$.
\end{proof}
\subsection{Самодвойственные функции.}
\begin{definition} Двойственной функцией к $f(\XXX)$ называется функция $f^*(x_1,\ldots,x_n)=\overline{f(\bar{x}_1,\ldots,\bar{x}_n)}.$ \end{definition}
\begin{example} $(x\&y)^*=x \vee y$ \end{example}

\noindent Легко заметить, что $(f^*)^*=f.$
\begin{definition}
Самодвойственная функция --- функция, двойственная самой себе; множество всех таких функций обозначается $S$.
\end{definition}
\begin{statement} 1) $\bar{x},x\oplus y \oplus z, m(x,y,z) \in S$; 2) $0,1,x\oplus y, x\rightarrow y, x \& y, x \vee y \notin S $ 
\end{statement}
\begin{proof}
	В этом несложно убедиться явной проверкой. 
\end{proof}	
\begin{lemma}
	$S$ является замкнутым классом.
\end{lemma}
\begin{proof}
	Тождественная функция содержится в ${S}$. Значит, осталось проверить, что если $f(x_1, \ldots,x_k), g_1, \ldots, g_k \in S$, то $h=f(g_1(\XXX),\ldots,g_k(\XXX)) \in S $. \\
	$h^*(x_1,\ldots,x_n)=\overline{f(g_1(\bar{x}_1,\ldots,\bar{x}_n)),\ldots,g_k(\bar{x}_1,\ldots,\bar{x}_n)}=\overline{f(\overline{\overline{g_1(\bar{x}_1,\ldots,\bar{x}_n)}},\ldots,\overline{\overline{g_k(\bar{x}_1,\ldots,\bar{x}_n)}}} = \\\overline{f(\overline{g^*_{1}({x_1},\ldots,{x_n})},\ldots,\overline{g^*_k({x_1},\ldots,{x_n})}} $. \\
	Так как $g_1=g^*_1,\ldots,g_k=g^*_k$, то \\$h=\overline{f(\overline{g_{1}({x_1},\ldots,{x_n})},\ldots,\overline{g_k({x_1},\ldots,{x_n})}}=f^*(g_1(x_1,\ldots, x_n),\ldots,g_k(x_1,\ldots,x_n)).\\ $
	$f^*=f$, значит 
	$\,h^*(x_1,\ldots,x_n)=f^*(x_1,\ldots,x_n)=f(x_1,\ldots,x_n)=$\\$=h(x_1,\ldots,x_n) \Rightarrow h \in S.$
\end{proof}
