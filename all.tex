\documentclass[12pt]{article}
\usepackage[utf8]{inputenc}
\usepackage[russian]{babel}
\usepackage[dvips]{graphicx}
\usepackage{pscyr}
\usepackage[T1]{fontenc}
\usepackage{amssymb, amsmath, textcomp, amsthm, multicol}
\usepackage{scalerel}
\textwidth=500pt
\textheight=750pt
\oddsidemargin=20pt
\hoffset=-1.5cm
\topmargin=-25mm

\usepackage{xcolor}
\usepackage{hyperref}
\definecolor{linkcolor}{HTML}{000000}
\definecolor{urlcolor}{HTML}{000000}
\hypersetup{pdfstartview=FitH, linkcolor=linkcolor, urlcolor=urlcolor, colorlinks=true}

\newtheoremstyle{neosn}{0.5\topsep}{0.5\topsep}{\rm}{}{\sc}{.}{ }{{\bf \thmname{#1}}\thmnote{ {\mdseries#3}}}
\newtheorem{theorem}{Теорема}
\newtheorem{lemma}{Лемма}
\theoremstyle{neosn}
\newtheorem{proposition}{Предложение}
\newtheorem{statement}{Утверждение}
\newtheorem{corollary}{Следствие}
\newtheorem{definition}{Определение}
\newtheorem{example}{Пример}
\renewcommand{\proofname}{Доказательство}
\newcommand{\XXX}{x_1, \ldots, x_n}
\newcommand{\SIG}{\sigma_1, \ldots, \sigma_n}


%команда для амперсанда, ага.
\DeclareMathOperator*{\amper}{\scalerel*{\&}{\sum}}
%\DeclareMathOperator*{\amper}{\&} % должно работать для старых дистрибутивов теха.

%правильные переносы
\newcommand*{\hm}[1]{#1\nobreak\discretionary{}%
{\hbox{$\mathsurround=0pt #1$}}{}}

%правильные больше-или-равно и меньше-или-равно
\renewcommand{\geq}{\geqslant}
\renewcommand{\leq}{\leqslant}

%\usepackage{dmvn}

\begin{document}
\section{Булевы функции}
\subsection{Определение булевой функции.}
Обозначим за $E$ множество $\lbrace0,1\rbrace$.

\begin{df}
$f(\XXX)\in{E}$ --- функция алгебры логики {\bf (булева функция)}, где $x_{i}\in{E} ~ \forall i=1,\ldots,n$ --- это отображение $f\colon E^{n}\rightarrow{E}$. Его можно проиллюстрировать таблицей возможных значений $f$ на различных наборах переменных:\\

$$\begin{array}{|ccccc|c|}
\hline
x_1 & x_2 & \ldots & x_{n-1} & x_n & f(\XXX)  \\
0 & 0 &\ldots & 0 & 1 & 0 ~ \textbf{или} ~ 1 \\
0 & 0 &\ldots & 1 & 1 & 0 ~ \textbf{или} ~ 1 \\
\ldots & \ldots & \ldots & \ldots & \ldots & \ldots\\
1 & 1 &\ldots & 1 & 1 & 0 ~ \textbf{или} ~ 1 \\
\hline
\end{array}$$
\\
\\\\
\end{df}

\begin{definition}
$P_{2}$ --- множество всех булевых функций от произвольного конечного множества переменных. $P_2(n)$ --- множество всех булевых функций от $n$ переменных. 
\end{definition}
\begin{definition}
$E^{n}=\{(\SIG)|~ \sigma_{i}\in E; ~ i=1,\ldots,n\}$ \\
\end{definition}
 
\begin{statement}
 $|P_2(n)|=2^{2^{n}}$. \\
\end{statement}
\begin{proof}
Очевидно.
\end{proof}
\subsection{Существенные и фиктивные переменные.} 
\begin{definition}
	Пусть $f(\XXX)$ --- булева функция. 
	Тогда $x_i$ называется \textbf{существенной} переменной для $f$, если:  $\exists{\sigma_1,\sigma_2, \ldots \sigma_{i-1}, \sigma_{i+1}, \ldots, \sigma_{n}}\in\{0,1\}$, такие, что: 

$ f(\sigma_1,\sigma_2, \ldots ,\sigma_{i-1}, 0, \sigma_{i+1}, \ldots, \sigma_{n})\neq f(\sigma_1,\sigma_2, \ldots, \sigma_{i-1}, 1, \sigma_{i+1}, \ldots, \sigma_{n}).$
	В противном случае переменная называется \textbf{фиктивной} (пример придумать не очень сложно).
\end{definition}
%\begin{enumerate}

1. Пусть $x_i$ --- фиктивная переменная для $f$. Рассмотрим функцию $g(x_1,x_2,\ldots,x_{i-1},x_{i+1}, \ldots, x_{n}),\\
    g(\sigma_1,\sigma_2 \ldots \sigma_{i-1}, \sigma_{i+1}, \ldots, \sigma_{n})=f(\sigma_1,\sigma_2 \ldots \sigma_{i-1},0, \sigma_{i+1}, \ldots, \sigma_{n}) =f(\sigma_1,\sigma_2 \ldots \sigma_{i-1},1, \sigma_{i+1}, \ldots, \sigma_{n}). $ Тогда говорят, что  \textbf{$g$ получена из $f$ удалением фиктивной переменной $x_i$}.

2. Пусть $f(\XXX)$ --- булева функция. Также, пусть имеется $y \neq \XXX $. Рассмотрим функцию $h(\XXX,y)$, $h(\SIG,\sigma)=f(\SIG)$. Тогда говорим, что \textbf{$h$ получена из $f$ добавлением фиктивной переменной  $y$.}

%\end{enumerate}
\begin{definition}
	Две булевы функции называются \textbf{равными}, если они могут быть получены друг из друга с помощью некоторого числа операций добавления или удаления фиктивных переменных. \\
\end{definition}
\subsection{Элементарные функции:} 
\begin{enumerate}
	\item От одной переменной.
	$$
    \begin{array}{|c|c|c|c|c|}
    \hline
    x & 0 & x & \bar{x} & 1 \\
    \hline
    0 & 0 & 0 & 1 & 1 \\
    \hline
    1 & 0 & 1 & 0 & 1 \\
    \hline
    \end{array}
    $$
    \item От двух переменных:
    $$
    \begin{array}{|c|c|c|c|c|c|c|c|c|c|}
	\hline
	 x & y & xy & x\vee y & x\oplus y & x\sim y & x\rightarrow y & x|y & x\downarrow y\\
	\hline
	 0 & 0 & 0 & 0 & 0 & 1 &  1 & 1 & 1 \\
	\hline
	 0 & 1 & 0 & 1 & 1 & 0 &  1 & 1 & 0 \\
	\hline
	 1 & 0 & 0 & 1 & 1 & 0 &  0 & 1 & 0  \\
	\hline
	 1 & 1 & 1 & 1 & 0 & 1 &  1 & 0 & 0 \\
	\hline
	\end{array}
	$$
	\item
	От трех переменных (функция "медиана"):
	$$
	\begin{array}{rrr|c}
	x~~ & y~~ & z~~ & f(x,y,z)\\
	\hline
	\begin{array}{r} % Вложенная таблица для каждого столбца
	0\\ 0\\ 0\\ 0\\ 1\\ 1\\ 1\\ 1\\
	\end{array}
	&
	\begin{array}{r}
	0\\ 0\\ 1\\ 1\\ 0\\ 0\\ 1\\ 1\\
	\end{array}
	&
	\begin{array}{r}
	0\\ 1\\ 0\\ 1\\ 0\\ 1\\ 0\\ 1\\
	\end{array}
	&
	\begin{array}{r}
	0\\ 0\\ 0\\ 1\\ 0\\ 1\\ 1\\ 1\\
	\end{array}
	\end{array}
	$$
\end{enumerate}	


\subsection{Формула над системой булевых функций.}
$\Phi=\{f_{1}(x_{1},x_{2},...,x_{n_1});f_{2}(x_{1},x_{2},...,x_{n_2});...;f_{n}(x_{1},x_{2},...,x_{n_n})\}\subseteq P_2$ --- некоторое множество булевых функций, таких что каждой булевой функции $f_{i}(x_{1},x_{2},...,x_{n_i})$ сопоставляем функциональный символ $f_{i}$.
\begin{definition}

	\textit{Формулой над $\Phi$} называется строка символов, состоящая из любых символов-переменных, обозначающих $f_1,...,f_n$ и вспомогательных символов $"("$,$")"$ ,$","$, определяемое индуктивным образом: 

\textbf{База индукции:} символ любой переменной --- правильная формула над $\Phi$.

\textbf{Индуктивное предположение: } пусть $F_1,F_2,...,F_{n_i}$ --- некоторые формулы над $\Phi$, тогда $f_i(F_1,F_2,...,F_{n_i})$ --- тоже формула над $\Phi$.

\end{definition}
\begin{example}
	$((\overline{x\vee y}) \& (z\rightarrow y ))$ --- формула над $\{x \vee y; x \& y, x \rightarrow y, \overline{x} \}$\\
\end{example}
Конъюнкция имеет приоритет над дизъюнкцией.\\

%\newpage 
%зачем тут newpage?
Значения формулы на наборе значений переменных, входящих в формулу, определяется индуктивным образом.

\textbf{База индукции:} если $f$ --- тривиальная, то все очевидно.

\textbf{Индуктивное предположение:} пусть $F_1,F_2, \ldots, F_n$ --- формулы, для которых данное понятие уже определено. \\
$F=f_i(F_1,F_2, \ldots, {F_{n_i}});$ \\
$\XXX$ --- все переменные, содержащиеся в F. \\
$\Omega=(\SIG)$ --- набор значений $\XXX$. \\
$\Omega_j$ --- поднабор значений из $\Omega$ для переменных, содержащихся в формуле $F_j$. \\
$b_j$ --- значение функции $F_j$ на наборе $\Omega_j$. \\
Тогда значение $F$ на наборе $\Omega$ равно $f_i(b_1, \ldots, {b_n}_i)$ \\
Пусть $F$ --- формула над $\Phi$, содержащая символы переменных $x_1, \ldots, x_n$. Тогда $F$ реализует функцию $f(\XXX)$, т.ч для любого набора $(\sigma_1, \ldots, \sigma_n)$ значений $x_1,...,x_n$ значение $f(\sigma_1, \ldots, \sigma_n)$ равно значению формулы $F$ на $\sigma_1, \ldots, \sigma_n$. \\
$f$ получается из $\Phi$ с помощью операции суперпозиции, если F реализуется некоторой нетривиальной формулой над $\Phi$.

\begin{definition}
Две формулы $F_1$ и $F_2$ называются {\bf эквивалентными}, если они реализуют одинаковые функции.
\end{definition}

Пусть $\ast \in \{\vee, \&, \oplus, \sim \}$ --- некоторая операция.

\begin{enumerate}
	\item 
		$x \ast y = y \ast x$ (коммутативность)
	\item 
		$x \ast (y \ast z) = (x \ast y) \ast z$ (ассоциативность)
	\item 
		$x (y \vee z) = xy \vee xz$

		$x (y \oplus z) = xy \oplus xz$

		$x \vee (y  \&  z) = (x \vee y) \& (x \vee z)$

		$x \vee (y  \sim  z) = (x \vee y) \sim (x \vee z)$ (дистрибутивность)
	\item
		$x \vee xy = x$ (поглощение)
	\item 
		$\overline{\overline{x}} = x$ (двойное отрицание)
	\item 
		$\overline{x \vee y} = \overline{x} \& \overline{y}$

		$\overline{x \& y} = \overline{x} \vee \overline{y}$ (закон де Моргана)
	\item 
		$x\overline{x} = 0$, \smallskip $x \vee \overline{x} = 1$, \smallskip $x \oplus \overline{x} = 1$, \smallskip $x \sim \overline{x} = 0$

		$xx = x$, \smallskip $x \vee x = x$, \smallskip $x \oplus x = 0$, \smallskip $x \sim x = 1$

		$x \& 1 = x$, \smallskip $x \vee 1 = 1$, \smallskip $x \oplus 1 = \overline{x}$, \smallskip $x \sim 1 = x$

		$x \& 0 = 0$, \smallskip $x \vee 0 = x$, \smallskip $x \oplus 0 = x$, \smallskip $x \sim 0 = \overline{x}$
\end{enumerate}

\section{Лекция 2 (Замыкания и прочее).}

\subsection{Определения.}

Возьмем множество $F \subseteq P_2$.

\begin{definition}
	Замыкание $[F]$ множества $F$ --- это множество всех булевых функций, получаемых из булевых функций множества $F$ с помощью операций суперпозиции, удаления и добавления фиктивных переменных.
\end{definition}

\begin{definition}
	$F$ --- замкнуто, если $[F] = F$.
\end{definition}

\begin{enumerate}
	\item
	$[\{x \oplus y\}] = \{0, x, x_1 \oplus \ldots \oplus x_t (t \ge 2)\}$
	\item
	$P_2$ --- замкнуто.	
\end{enumerate}

\begin{definition}
	$P_2(n)$ --- все булевы функции, существенно зависящие от не более, чем $n$ переменных.
\end{definition}

\begin{enumerate}
	\item
	$P_2(1)$ --- замкнуто.
	\item
	$P_2(2)$ --- не замкнуто. $\left( xy \in P_2(2), xyz \not\in P_2(2) \right)$
\end{enumerate}
\subsection{Свойства замыкания.}
\begin{enumerate}
	\item $F \subseteq [F].$
	\item $F_1 \subseteq F_2 \Longrightarrow [F_1] \subseteq [F_2]$
	\item $[[F]] = [F]$
	\begin{proof}
		1) $[F] \subseteq [[F]]$ (по 1, 2)

		2)$[[F]] \subseteq [F]$.

		$f(\XXX) \in [[F]] \Rightarrow \exists$ формула $\Phi$, реализующая $f$. Пусть $f_1, \ldots, f_s$ --- все функциональные символы, содержащиеся в $\Phi$. $f_1, \ldots, f_s \in [F] \Rightarrow $ каждая функция $f_i$ реализуется некоторой формулой $\Phi_i$ над $F$ : $\Phi = f_i(F_1, \ldots, F_{n_i})$.

		$\Phi_i(F_1, \ldots, F_{n_i})$ --- формула, полученная из $\Phi$ заменой $x_i \longmapsto F_i$. $\Phi_i(F_1, \ldots, F_n).$

		$\Phi_i(F_1, \ldots, F_n).$

		Так получим: 

		$\Phi'$ --- формулу над $F$, реализующую функцию $F \Rightarrow f \in [F] \Rightarrow [[F]] \subseteq [F]$.
	\end{proof}
	\item  $[F_1] \cap [F_2]$ -- замкнуто.
	\begin{proof}
		Возьмем $f \in [[F_1] \cap [F_2]]$: $f$ реализуется формулой $\Phi$ над $[F_1] \cap [F_2]$. Пусть $f_1, \ldots f_s$ -- все функциональные символы из $\Phi$. $\forall \: i \: f_i$ реализуется и формулой $\Phi_1$ над $F_1$ и формулой $\Phi_2$ над $F_2 \Rightarrow f \in  [F_1] \cap [F_2]$.
	\end{proof}
	\item  $[F_1] \cup [F_2]$ не обязательно замкнуто.
\end{enumerate}

\subsection{}
Пусть $F$ --- замкнутое множество, и $F_1 \subseteq F$.

\begin{definition}
	$F_1$ называется полным в $F$, если $[F_1] = F$.
\end{definition}

\begin{definition}
	$F_1$ называется полным, если $[F_1] = P_2$.
\end{definition}

\begin{example}
	$P_2$ -- полное множество.
\end{example}

\begin{statement}
	$f(\XXX)$ -- булева функция. Тогда: $f(\XXX) = (\overline{x_1} \: \& \: f(0, x_2, \ldots, x_n)) \vee (x_1 \: \& \: f(1, x_2, \ldots , x_n))$
\end{statement}
\begin{proof}
	Пусть $\sigma = (\SIG) $ --- набор значений $\XXX$.
\begin{enumerate}
\item $\sigma_1 = 0$. 

$\overline{\sigma_1} \: \& \: f(0, \sigma_2, \ldots, \sigma_n) \vee \sigma_1 \: \& \: f(1, \sigma_2, \ldots, \sigma_n) = 1 \: \& \: f(0, \sigma_2, \ldots, \sigma_n) \vee 0 \: \& \: f(1, \sigma_2, \ldots, \sigma_n) = f(0, \sigma_2, \ldots, \sigma_n) = f(\SIG) $

\item $\sigma_1 = 1$.

$0 \: \& \: f(0, \sigma_2, \ldots, \sigma_n) \vee 1 \: \& \: f(1, \sigma_2, \ldots, \sigma_n) = f(1, \sigma_2, \ldots, \sigma_n) = f(\SIG) $
\end{enumerate}
\end{proof}

$
f(\XXX) = (\overline{x_1} \: \& \: f(0, x_2, \ldots, x_n)) \vee (x_1 \: \& \: f(1, x_2, \ldots , x_n)) = $

$= \overline{x_1} \: \& \: (\overline{x_2} \: \& \: f(0, 0, \ldots, x_n)) \vee (x_2 \: \& \: f(0, 1, \ldots , x_n)) \vee (x_1 \: \& \: (\overline{x_2} \: \& \: f(1, 0, \ldots, x_n)) \vee (x_1 \: \& \: f(1, 1, \ldots , x_n))) =$

$
\overline{x_1}\overline{x_2}f(0, 0, \ldots, x_n) \vee \overline{x_1} x_2 f(0, 1, \ldots, x_n) \vee x_1 \overline{x_2}f(1, 0, \ldots, x_n) \vee x_1 x_2 f(1, 1, \ldots, x_n)
$

\begin{definition}
	$x_\sigma = \begin{cases} x \text{, если }\sigma = 1 \\ \overline{x} \text{, если }\sigma = 0 \end{cases}$
\end{definition}


Итак, $f(\XXX)$ можно переписать в виде $\bigvee \limits_{\sigma_1, \sigma_2 \in E} f(\sigma_1, \sigma_2, x_3, \ldots, x_n)$.

Мы также можем разложить $f$ по $k$ переменным:

$f(\XXX) = \bigvee \limits_{(\sigma_1,\ldots, \sigma_k) \in E^k} f(\sigma_1, \ldots, \sigma_k, \ldots, x_n)$

При $k = n$ получаем:  $f(\XXX) = \bigvee \limits_{(\sigma_1,\ldots, \sigma_n) \in E^n} x_1^{\sigma_1}\ldots x_n^{\sigma_n}f(\SIG) =$

$ = \bigvee \limits_{\substack{(\sigma_1,\ldots, \sigma_n) \in E^n \\ f(\SIG) = 1}} x_1^{\sigma_1}\ldots x_n^{\sigma_n} $ -- {\large Совершенная дизъюнктивная нормальная форма ({\bf СДНФ}).}

\begin{statement}
$	\{x\&y, x \vee y, \bar{x} \} - \text{полное множество.} $
\end{statement}
\begin{proof}
    Если $f \ne 0$, то СДНФ - формула над $\{x\&y, x \vee y, \bar{x} \}$
    Если $f = 0$, то $f = \overline{x} \& x \Rightarrow $ любая функция реализуется формулой над $\{x\&y, x \vee y, \bar{x} \}$.
\end{proof}

\begin{lemma}[О сводимости полных множеств.]
	$F, F' \subseteq P_2$, $F$ -- полное множество и любая функция из $F$ может быть реализована формулой над $F' \Rightarrow F'$ -- полное множество.
\end{lemma}
\begin{proof}
	$\forall$ функция из $F$ может быть реализована формулой над $F' \Rightarrow F \subseteq [F'] \Rightarrow [F] \subseteq [[F']] = [F'].$

	$F$ -- полное $\Rightarrow [F] = P_2, [F] \subseteq [F'] \Rightarrow P_2 \subseteq [F'] \Rightarrow F'$ -- полное.
\end{proof}

\newpage

\begin{statement}
$	\{x\&y, \bar{x} \} - \text{полное множество.} $
\end{statement}
\begin{proof}
			$\{x\vee y, x \& y, \bar{x} \}  - \text{полное множество. Учитывая, что: } 
			x \vee y = \overline{\bar{x}\&\bar{y} } $ , то по лемме о сходимости получаем нужное. 

\end{proof}	
\begin{statement}
	$\{x \vee y, \bar{x} \}$ -- полное множество.  
\end{statement}
\begin{proof}
	$	\{x\&y, \bar{x} \} - \text{полное множество.} $ Учитывая, что: $x \& y =  \overline{\bar{x}\vee\bar{y} }$, то по лемме о сходимости получаем нужное.
\end{proof}
\newpage
\begin{statement} 
$\{x \oplus y; x \& y, 1\}$ -- полное множество.
\end{statement}
\begin{proof}
	$\bar{x}=x \oplus 1. $ Получаем нужное по лемме о сходимости и утверждению 2.
\end{proof}	
\begin{statement}
	$\{x | y\}$ -- полное множество. 
\end{statement}
\begin{proof}
	$x|y=\bar{x} \vee \bar{y}=\overline{x \& y}$ \\
	$\bar{x}=x | x. \\
	x \& y=\overline{x|y}=(x|y)|(x|y) \\
	{x \& y, \bar{x}} \text{ -- полное по лемме о сходимости.} \Rightarrow {x|y}$ - полное.   
\end{proof}
\begin{corollary}
	Из любого полного множества можно выделить конечное полное подмножество.
\end{corollary}
\begin{proof}
	$F\subseteq P_2$ -- полное множество. $\Rightarrow$ существует формула Ф над $F$, реализующая ${x|y}$. Пусть $\{f_1, \ldots, f_s\}$ - множество всех символов функций, содержащихся в Ф.Ф -- формула над $\{f_1, \ldots, f_s\}$ $\Rightarrow  ~ x|y$ содержится в замыкании. $\{x|y\}$ -- полное. $\Rightarrow$ по лемме о сводимости $\{f_1, \ldots, f_s\} \subseteq F$ -полное.
\end{proof}  
\begin{document}

\begin{statement}
$	\{x\&y, \bar{x} \} - \text{полное множество.} $
\end{statement}
\begin{proof}
			$\{x\vee y, x \& y, \bar{x} \}  - \text{полное множество. Учитывая, что: } 
			x \vee y = \overline{\bar{x}\&\bar{y} } $ , то по лемме о сходимости получаем нужное. 

\end{proof}	
\begin{statement}
	$\{x \vee y, \bar{x} \}$ -- полное множество.  
\end{statement}
\begin{proof}
	$	\{x\&y, \bar{x} \} - \text{полное множество.} $ Учитывая, что: $x \& y =  \overline{\bar{x}\vee\bar{y} }$, то по лемме о сходимости получаем нужное.
\end{proof}
\begin{statement} 
$\{x \oplus y; x \& y, 1\}$ -- полное множество.
\end{statement}
\begin{proof}
	$\bar{x}=x \oplus 1. $ Получаем нужное по лемме о сходимости и утверждению 2.
\end{proof}	
\begin{statement}
	$\{x | y\}$ -- полное множество. 
\end{statement}
\begin{proof}
	$x|y=\bar{x} \vee \bar{y}=\overline{x \& y}$ \\
	$\bar{x}=x | x. \\
	x \& y=\overline{x|y}=(x|y)|(x|y) \\
	{x \& y, \bar{x}} \text{ -- полное по лемме о сходимости.} \Rightarrow {x|y}$ - полное.   
\end{proof}
\begin{corollary}
	Из любого полного множетсва можно выделить конечное полное подмножество.
\end{corollary}
\begin{proof}
	$F\subseteq P_2$ -- полное множество. $\Rightarrow$ существует формула Ф над $F$, реализующая ${x|y}$. Пусть $\{f_1, \ldots, f_s\}$ - множество всех символов функций, содержащихся в Ф.Ф -- формула над $\{f_1, \ldots, f_s\}$ $\Rightarrow  ~ x|y$ содержится в замыкании. $\{x|y\}$ -- полное. $\Rightarrow$ по лемме о сводимости $\{f_1, \ldots, f_s\} \subseteq F$ -полное.
\end{proof}  
\subsection{Полином Жегалкина.} 
$f(x_1, \ldots, x_n)$ -- булева функция. $f\neq 0$. \\
$f(x_1, \ldots, x_n)=\bigvee \limits_{f(\sigma_1, \ldots, \sigma_n)=1} x^{\sigma_1}, \ldots, x^{\sigma_n}=\bigoplus \limits_{f(\sigma_1, \ldots, \sigma_n)=1} x^{\sigma_1}, \ldots, x^{\sigma_n}=$ 
\begin{flushright}
$
=\bigoplus \limits_{f(\sigma_1, \ldots, \sigma_n)=1} (x_1\oplus \bar{\sigma_1}),\ldots,(x_n\oplus \bar{\sigma_n})=\bigoplus \limits_{i_1<i_2< \ldots,<i_k} c_{i_1, \ldots, i_k}x_{i_1}, \ldots ,x_{i_k} \bigoplus C $, где  $c_{i_1, \ldots, i_k}=\{0,1\}$.
\end{flushright} 
$0$ - полином Жегалкина для $f=0$.(по определению) 

\begin{statement}
	Любая булева функция может быть реализована единственным полиномом Жегалкина.(с точностью до перестановки слагаемых и сомножителей). 
\end{statement}
\begin{proof}
	Полином Жегалкина единственен, так как $c_{i_1}, \ldots, c_{i_n}$ -- однозначно определены $2^n$ -  коэффицентов $2^{2^n}$ - набор значений коэффицентов $2^{2^n}$ - количество функций от $n$ - переменных $\Rightarrow$ для каждой функции $\exists !$ полином Жегалкина. 
\end{proof}
\subsection{Функции, сохраняющие ноль и единицу. \\}
\begin{definition}
	$f$ - сохраняет 0, если $f(0,\ldots, 0)=0$.
	$T_0$ - множество всех функций, сохраняющих ноль.(например, $0,x,x \& y, x \vee y, x \oplus y)$.
\end{definition}
\begin{definition}
	Селекторная функция - функция, тождественна равная переменной. \\
\end{definition}
\begin{lemma}
	$T_0 $ - замкнуто. 
\end{lemma}
\begin{proof}
	Тождественная функция содержится в $T_0$.Значит, надо проверить, что если  \\
	 $g(x_1,\ldots, x_n),g_1, \ldots, g_n \in T_0, \text{,то~} f(g_1, \ldots, g_n) \in T_0. \\$
	Можем полагать, что $g_1, \ldots, g_n$ зависят от одних и тех же переменных: $x_1, \ldots, x_n$(иначе можно добавить переменные в качестве фиктивных).Тогда:\\
	$f(g_1(x_1, \ldots, x_n),\ldots, g_n(x_1, \ldots, x_n))=h(x_1, \ldots, x_n) \\
	h(0, \ldots, 0)=f(g_1(0,\ldots,0),\ldots,g_n(0, \ldots, 0))=f(0,\ldots,0)=0. \Rightarrow h \in T_0. 
$
\end{proof}
\begin{definition}
	$f$ - сохраняет 1, если $f(1,\ldots, 1)=1$.
	$T_1$ - множество всех функций, сохраняющих ноль.(например, $(1,x,xy,x\rightarrow y, x \vee y)$
\end{definition}
\begin{lemma}
	$T_1 $ является замкнутым классом. 
\end{lemma}
\begin{proof}
	Аналогично предыдущей лемме. 
\end{proof}
\subsection{Монотонные функции.}	
Определим правило сравнения на наборах из нулей и единиц.\\
$\sigma'=\{\sigma_1',\ldots,\sigma_n'\}, \sigma''= \{\sigma_1'',\ldots,\sigma_n''\} \in \{0,1\}^n.$ \\
Будем говорить, что $\sigma'<\sigma'', \text{если} ~ \sigma_i ' < \sigma_i '' ~ \forall i.$ \\
Заметим, что существуют несравнимые наборы, например: (101) и (010). 
\begin{definition}
$f$ --монотонная, если для любых $\sigma' \text{ и } \sigma'' ~ \text{таких, что:} ~ \sigma'\leq \sigma'' $выполняется, что:$ f(\sigma')\leq f(\sigma''). $
\end{definition}
\begin{lemma}
	M является замкнутым классом. 
\end{lemma}
\begin{proof}
	%Воткнуть в обозначения лекций я так и не смог, потому частично взял с Лупанова. 
Тождественная функция содержится в M. Значит, осталось проверить, что если
$f(x_1, \ldots,x_k), g_1, \ldots, g_k \in M$, то $h=f(g_1,\ldots,g_n) \in M $. Можно считать, что  $g_1,\ldots g_n$  -- функции от одного и того же количества переменных, в противном случае недостающие переменные можно добавить в качестве несущественных.  Выберем произвольные различные наборы $ 
\sigma'=\{\sigma_1',\ldots,\sigma_n'\}, \sigma''= \{\sigma_1'',\ldots,\sigma_n''\}$, такие что: $ \sigma' \leq \sigma''.\\
$Рассмотим $ h(\sigma')=f(g_1(\sigma'),g_2(\sigma'), \ldots, g_k(\sigma')) ~\text{и}~ h(\sigma'')=f(g_1(\sigma''),\ldots,g_k(\sigma'')). \\
g_i(\sigma')<g_i(\sigma''), $ так как $ g_i$ - монотонная. $ f(g_1(\sigma'),g_2(\sigma'), \ldots, g_k(\sigma')) \leq f(g_1(\sigma''),g_2(\sigma''), \ldots, g_k(\sigma''))$, так как f- монотонная.$\Rightarrow h $ - тоже монотонная. 
\end{proof}
\begin{lemma}[О немонотонных функциях]
	$f(x_1,\ldots, x_n)\notin M. $ Тогда $\bar{x} \in [\{f;0;1\}].$
\end{lemma}	
\begin{proof}
	$f \notin M \Rightarrow \exists \sigma' \text{и} ~ \sigma''$,такие что $ \sigma \leq \sigma''~ \text{и}~ f(\sigma')=1, f(\sigma'')=0, \text{где }  \sigma',  \sigma'' \text{-разные}. \\
$Без ограничения общности будем считать, что $\sigma' ~\text{и} ~ \sigma''$ устроены следующим образом: \\$ 
	\sigma'=(0,\ldots,0,\ldots,0,1,\ldots,1) \\
	\sigma''=(\underbrace{1,\ldots,1}_k,\underbrace{0,\ldots,0}_s,\underbrace{1,\ldots,1}_{n-k-s}) \\
	g(x)=f(\underbrace{x, \ldots, x}_k, \underbrace{0, \ldots, 0}_s,\underbrace{1,\ldots,1}_{n-k-s}) =\bar{x}.$ Так как, ($g(0)=1,g(1)=0$).
\end{proof}
\subsection{Самодвойственные функции.}
	$f^*(x_1,\ldots,x_n)=\overline{f(\bar{x_1},\ldots,\bar{x_n})}.$ $f^*$ называется двойственной функцией к $f$.($(x\&y)^*=x \vee y$)	
	Легко заметить, что $(f^*)^*=f.$
	\begin{definition}
		Самодвойственная функция -- функция, двойственная сама себе, множество всех таких функций обозначается $S$.
	\end{definition}
\begin{statement}
	$\bar{x},x\oplus y \oplus z, m(x,y,z) \in S; ~
	0,1,x\oplus y, x\textrightarrow y, x \& y, x \vee y \notin S $ 
\end{statement}
\begin{proof}
	Явной проверкой в этом несложно убедиться. 
\end{proof}	
\begin{lemma}
	$S$ является замкнутым классом.
\end{lemma}
\begin{proof}
	Тождественная функция содержится в S. Значит, осталось проверить, что если $f(x_1, \ldots,x_k), g_1, \ldots, g_k \in M$, то $h=f(g_1,\ldots,g_k) \in M $. \\
	$h^*(x_1,\ldots,x_n)=\overline{f(g_1(\bar{x_1},\ldots,\bar{x_n})),\ldots,g_k(\bar{x_1},\ldots,\bar{x_n})}=\overline{f(\overline{\overline{g_1(\bar{x_1},\ldots,\bar{x_n})}},\ldots,\overline{\overline{g_k(\bar{x_1},\ldots,\bar{x_n})}}}; \\\overline{f(\overline{g^*_{1}({x_1},\ldots,{x_n})},\ldots,\overline{g^*_k({x_1},\ldots,{x_n})}} $ \\
	Так как $g_1,\ldots,g_k \in S \Rightarrow g_1=g^*_1;\ldots;g_k=g^*_k$, то: \\
	$h=\overline{f(\overline{g_{1}({x_1},\ldots,{x_n})},\ldots,\overline{g_k({x_1},\ldots,{x_n})}}=f^*(g_1(x_1,\ldots, x_n),\ldots,g_k(x_1,\ldots,x_n)).\\ $
	$f \in S \Rightarrow f^*=f$, значит \\
	$h^*(x_1,\ldots,x_n)=f^*(x_1,\ldots,x_n)=f(x_1,\ldots,x_n)=h(x_1,\ldots,x_n). \Rightarrow h \in S.$
\end{proof}
\end{document}
\begin{lemma} [О несамодвойственной функции] Пусть $f(\XXX)\not\in S$, тогда $0,\,1 \in [\{f, \bar{x}\}]$
\end{lemma}
\begin{proof} 
    Пусть $f(\XXX)\not\in S$, тогда  $f^*(\XXX) = \overline{f(\bar{x}_1,\ldots,\bar{x}_n)} \neq f(\XXX) \Rightarrow \exists \sigma = (\sigma_1,\ldots,\sigma_n),$ т.ч. $\overline{f(\bar{\sigma}_1,\ldots,\bar{\sigma}_n)} \neq f(\sigma_1,\ldots,\sigma_n) \Rightarrow f(\bar{\sigma}_1,\ldots,\bar{\sigma}_n) = f(\sigma_1,\ldots,\sigma_n) = C.$\\
    Будем считать, что $(\SIG) = (\underbrace{0,\ldots,0}_k,\underbrace{1,\ldots,1}_{n-k})$.\\
    Пусть $g(x) = f(\underbrace{x,\ldots,x}_k,\underbrace{ \bar{x},\ldots,\bar{x}}_{n-k}).$\\
    $g(0) = f(\underbrace{0,\ldots,0}_k,\underbrace{ 1,\ldots,1}_{n-k}) = f(\SIG),\\
    g(1)= f(\underbrace{1,\ldots,1}_k,\underbrace{ 0,\ldots,0}_{n-k}) = f(\bar{\sigma}_1,\ldots,\bar{\sigma}_n)$,\\ значит, 
    $g(0) = g(1) = C, $ причём $g$ задаётся формулой над $\{f,\,\bar{x}\} \Rightarrow g \in S$.\\
    Получаем $C \in  [\{f, \bar{x}\}] \Rightarrow \bar{C} \in [\{f, \bar{x}\}] \Rightarrow 0,\,1 \in [\{f, \bar{x}\}].$
\end{proof}

\subsection{Линейные функции.}
\begin{definition}
Булева функция называется линейной, если степень её полинома Жегалкина не превосходит $1$.
\end{definition}
Здесь под степенью полинома Жегалкина понимается максимальная длина слагаемого в нём или, говоря алгебраическим языком, его степень как многочлена над $\mathbb{Z}_2$. Например, степень полинома $xyz \oplus x \oplus 1$ равна $3$. 
\begin{definition}
$L$ --- класс всех линейных булевых функций.
\end{definition}
\begin{proposition}
1) $0,\, 1,\, x,\, \bar{x},\, x\oplus y,\, x \sim y \in L$, 2) $x \to y,\, x \vee y,\, x\& y \not\in L.$
\end{proposition}
\begin{proof}
Первая часть утверждения очевидна, кроме утверждения про функцию $x \sim y$. Чтобы доказать оставшееся, представим следующие функции в виде полиномов:\\
$x \sim y = x \oplus y \oplus 1 ~ (\deg = 1),\\
x \to y = \bar{x} \vee xy = xy \oplus x \oplus 1 ~ (\deg = 2),\\
x \vee y = xy \oplus x \oplus y ~ (\deg = 2)$.
%дописать про последнюю функцию x\& y !!!
\end{proof}
\begin{lemma}
$L$ является замкнутым классом.
\end{lemma}
\begin{proof}
$x \in L$. Достаточно доказать, что $f(x_1, \ldots, x_k),\, g_1(\XXX),\, \ldots,\,\\ g_k(\XXX) \in L \Rightarrow h(\XXX) = f(g_1(\XXX),\ldots,g_k(\XXX)) \in L $.\\
Проверим это напрямую:\\
$f\in L \Rightarrow f(x_1,\ldots,x_k) = c_1x_1 \oplus \ldots \oplus c_kx_k \oplus c; ~ c_i,c 
\in\{0,1\}.$\\
$g_1, \ldots, g_k \in L \Rightarrow g_i(\XXX) = d_{i1}x_1 \oplus \ldots \oplus d_{in}x_n \oplus d_i; ~ d_{ij},d_i \in \{0,1\}. $\\
$h(\XXX) = c_1\big(d_{11}x_1 \oplus \ldots \oplus d_{1n}x_n \oplus d_1 \big) \oplus \ldots \oplus c_k\big(d_{k1}x_1 \oplus \ldots \oplus d_{kn}x_n \oplus d_k \big) \oplus c = \\ = (c_1d_{11} \oplus c_kd_{k1})x_1 \oplus \ldots \oplus (c_1d_{1n} \oplus \ldots \oplus c_kd_{kn})x_n \oplus (c_1d_1 \oplus \ldots c_kd_k \oplus c).$ Видно, что это линейная функция.
\end{proof}
\begin{lemma}[О нелинейной функции] Пусть $f(\XXX)\not\in L$. Тогда $x\& y \in [\{f, \bar{x}, 0, 1\}].$
\end{lemma}
\begin{proof} Пусть $f\not\in L$, тогда степень её полинома Жегалкина равна $k \geq 2$. Выберем нелинейное слагаемое наименьшей степени $l\geq 2$ в этом полиноме. Без ограничения общности можно считать, что это слагаемое $x_1...x_l$. Запишем $f$ в виде $f(x_1,\ldots,x_n) = f_{\deg > l} \oplus x_1...x_l \oplus f_{\deg \leq 1}$, где $f_{\deg > l}$ --- сумма всех слагаемых степени больше $l$, а $f_{\deg \leq 1}$ --- сумма всех слагаемых степени не больше $1$.

Рассмотрим функцию $g(x,y) = f(x, \overbrace{y,...,y}^{l-1}, 0,...,0)$. Ясно, что при подстановке аргументов $(x, y,...,y, 0,...,0)$ в полином Жегалкина для $f$ занулятся все слагаемые, входящие в $f_{\deg > l}$. 
Далее, $g(x,y) = x\underbrace{y...y}_{l-1} \oplus \ldots = xy \oplus c_1x \oplus c_2y \oplus c$. 

Теперь рассмотрим функцию $g'(x,y) = g(x\oplus c_2, y\oplus c_1) = xy \oplus c_1c_2 \oplus c = xy \oplus d$. Значит, $xy = g'(x,y) \oplus d = g(x\oplus c_2, y\oplus c_1) \oplus d = f(x \oplus c_2, \underbrace{y\oplus c_1,\ldots,y\oplus c_1}_{l-1},0,\ldots,0)\oplus d$. 

Так как %$x\oplus d = \left\{ \begin{gathered} x,\,d=0\hfill \\ \bar{x},\,d=1,\hfill \\ \end{gathered} \right.$ то 
$x\oplus d = x$ при $d=0$ и $x\oplus d = \bar{x}$ при $d=1$, то $xy \in [\{f, \bar{x}, 0, 1\}].$
\end{proof}

\subsection{Критерий Поста.}
\begin{theorem}[Критерий полноты] Пусть $\mathcal{F} \subseteq P_2$, тогда \\ $\mathcal{F}$ является полной в $P_2$ $\Longleftrightarrow$ $\mathcal{F}$ не содержится ни в одном из классов $ T_0,\, T_1,\,M,\,L,\,S.$
\end{theorem}
\begin{proof}
~\\
1. $(\Rightarrow)$. Пусть $X$ --- один из классов $ T_0,\, T_1,\,M,\,L,\,S$. Они замкнуты, то есть $[X] = X$.
Предположим, $\mathcal{F} \subseteq X $, тогда $[\mathcal{F}] \subseteq [X] = X \neq P_2 $. Противоречие. Значит, $\mathcal{F}$ не содержится ни в одном из классов $ T_0,\, T_1,\,M,\,L,\,S.$
~\\
2. $(\Leftarrow)$. Пусть $\mathcal{F}$ не содержится ни в одном из классов $ T_0,\, T_1,\,M,\,L,\,S$. Тогда существуют функции $f_X \in \mathcal{F} \setminus X$, где $X \in \{T_0,\, T_1,\,M,\,L,\,S\}$. 

Получим из этих функций константы, отрицание и дизъюнкцию. $f_{T_0} \not\in T_0 \Rightarrow f_{T_0} (0,\ldots,0) = 1 $. 


%$f_{M} $,\\
%$ $,\\
%$ $.\\
\end{proof} % надо дописать!
\subsection{Предполные классы}

\begin{definition}
	Пусть $F \subseteq P_2$. Тогда $F$ -- предполный, если: 
	\begin{enumerate}
		\item $F \ne P_2$
		\item $[F] = F$
		\item $\forall f \notin F: [F \cup {f}] = P_2$
	\end{enumerate}
\end{definition}

В клетке таблице снизу стоит функция $\in$ строке и $\not \in $ столбцу $\Rightarrow$ ни один из классов не содержится в другом.

$$\begin{array}{|c|c|c|c|c|c|}
\hline
 & T_0 & T_1 & L & S & M  \\
\hline
T_0 &  & 0 & xy & xy & x \oplus y \\
\hline
T_1 & 1 & &  xy & xy &  x ~ y \\
\hline
L & \overline{x} & \overline{x} &  & x \oplus y & \overline{x}\\
\hline
S & \overline{x} & \overline{x} & m(x, y, z) &  & \overline{x} \\
\hline
M & 1 & 0 & xy & xy &  \\
\hline
\end{array}$$
\\
\\\\

\begin{theorem}
	$T_0, T_1, S, M, L$ -- множеcтво всех предполных классов.
\end{theorem}

\begin{proof}
	Пусть есть $A = [A]$ -- предполный. Возьмем класс $M$.

	Пусть $A \subset M \Rightarrow \exists f \in M \setminus A : [A \cup {f}] \subseteq [M] = M$.

	Аналогичные рассуждения можно провести и для остальных классов.
\end{proof}

\subsection{Принцип двойственности}
{\bf Формулировка.}

Пусть формула $\Phi$ над $F$ задает функцию $f$. Формула $\Phi'$, получающаяся из $\Phi$ путем замены $f_i \rightarrow f_i^{*}$ будет реализовывать $f^{*}$.

$\left( \text{Напомним, что функция } f^{*}(\XXX) = \overline{f(\overline{x_1}, \ldots, \overline{x_n})} \right)$

{\bf Идея: }


$ \overline{f_0(  \overline{\overline{f_{i_1}(\ldots)}}, \ldots, \overline{\overline{ f_{i_k}(\ldots)}} )} $ -- возникаю двойные отрицания, которые изчезают, кроме {\em верхних} и {\em нижних}, что как раз дает функцию $f^{*}$.


\section{Сложность.}

Последовательность элементарных операций из заданного множества операций $F$ над заданным множеством входных данных $X$.

\begin{ex}
	$X = \{x\}, F = \{x\}$

	Мы спокойно можем вычислить $x_2, x_4, x_8, x-{16}, x_{32} \ldots$

	Вычислить $x_n$ мы можем на $\leq 2\log_2(n)$, однако нам как минимум понадобится $\log_2(n)$ операций.
\end{ex}

\begin{exer}
	Для вычисления $x^n$ достаточно $\log_2(n) + O(\log_2(n))$ операций.
\end{exer}

\section{Cхемы} 

\subsection{Отступление. Графы}

Обозначим за $V = \{v_1, \ldots, v_n\}$ -- множество "вершин" нашего графа, а за $E = \{e_1, \ldots, e_k\}$ -- множество "ребер". 

$\rho : E \rightarrow V^1 \cup V^2$ (где $V^i$ -- множество всех $i-$элементных подмножеств множества $V$).

Возьмем $e \in E$. Если

\begin{itemize}
	\item $\rho(e) \in V^1$ , то $e$ -- петля.
	\item $\rho(e_1) = \rho(e_2)$ , то $e_1$ и $e_2$ -- кратные ребра.
	\item $\rho(e) = \{v_1, v_2\}$, то $v_1$ и $v_2$ -- инцидентны.
\end{itemize}

\begin{df}
	$deg(v)$ -- число ребер, инцидентных $v$ (петля дает 2).
\end{df}

\begin{itemize}
	\item Если $deg(v) = 0$, то $v$ -- изолированная вершина
	\item Если $deg(v) = 1$, то $v$ -- висячая(концевая) вершина
\end{itemize}

\begin{df}
 	\begin{itemize}
 		\item $e_1, e_2, \ldots, e_{k-1}$ -- {\bf путь}	 из $v_{i_1}$ в $v_{i_k}$, если $\rho(e_1) = \{v_{i_1}, v_{i_2}\}, \rho(e_2) = \{v_{i_2}, v_{i_2}\} , \ldots, \rho(e_{k-1}) = \{v_{i_{k-1}}, v_{i_k}\}$
 		\item Если все $v_{i_k}$ различны, то путь называется {\bf цепью}
 		\item Путь, в котором $v_{i_1} = v_{i_k}$ называется {\bf циклом}
 		\item Цикл называется простым, если в нем все вершины и ребра различны, кроме $v_{i_1} = v_{i_k}$
 	\end{itemize}
\end{df}

\begin{df}
	Граф $G = (V, E, \rho)$ называется {\bf связным}, если для любых вершин $v_1$ и $v_2$ существует путь из $v_1$ в $v_2$.

	Дерево -- связный граф без простых циклов.
\end{df}

\begin{exer}
	$G = (V, E, \rho)$ -- дерево $\Longleftrightarrow$
	\begin{enumerate}
		\item $G$ -- связен, но после удаления любого ребра -- не связен
		\item $G$ -- без простых циклов, но при добавлении любого ребра -- есть цикл
		\item Любые двее вершины соединениы единственной простой цепью
		\item $G$ -- связен, $|E| = |V| - 1$
		\item В $G$ нет простых циклов, $|E| = |V| - 1$
	\end{enumerate}
\end{exer}

\begin{df}
	Ориентированный граф -- то же самое, только $\rho: E \rightarrow V \times V$
\end{df}


\section{Схемы из функциональных элементов}

$B_0 = \{x \& y, x \vee y, \overline{x}\}$  

$X = \{\XXX\}$

СФЭ в $B_0$ -- это ориентированный граф без ориентированных циклов такой, что входящая степень каждой вершины может быть только 0, 1 или 2.

Если $deg^{+}(v) = 0$, то символ из $X$, если $deg^{+}(v) = 1$, то операция 'отрицание', если $deg^{+}(v) = 2$, то $\&$ или $\vee$.

Кроме того, одна или несколько вершин помечены $\bigstar$  -- это "выходы".

Если $deg(v) = 0$, то назовем $v$ входом.

\begin{stm}
	В любом ориентированном графе без ориентированных циклов есть $v \in V$, такая что $deg(v) = 0$
\end{stm}

\begin{stm}
	В любом ориентированном графе без ориентированных циклов с $n$ вершинами можно занумеровать вершины от 1 до $n$ так, что все ребра ведут от вершины с меньшим номером к вершине с большим номером.
\end{stm}

(*здесь должен быть пример с картинкой*)


СФЭ в базие $B$ над $X$ -- последовательность равенств

$y_1 = \varphi_1(z_{1,1}, \ldots, z_{1,r_1})$ 

$ \ldots$

$y_i = \varphi_i(z_{i,1}, \ldots, z_{i,r_i})$ 

$\ldots$

$ y_l = \varphi_l(z_{l,1}, \ldots, z_{l,r_l})$

Все $\varphi_i \in B$ и $z_{i,j} \in X \cup \{y_1, \ldots, y_{i - 1}\}$

(*Здесь тоже не хватает картинки*)

\begin{df}
	Сложность схемы -- число вершин, не являющихся входом.
\end{df} % пока с середины
%\include{lec6} %ещё не написана
% это 7ая лекция, кажется.
\section{Лекция 7 (продолжение оценки функции Шеннона).} 
\subsection{В предыдущих сериях}
Мы уже имеем для функции Шеннона следующую оценку скорости роста:
%Надо заменить значки меньше или равно на меньше или эквивалентно
$$n\leq L(n)\leq 6\cdot \frac{2^n}{n}$$
Попробуем теперь её улучшить.\\
Напомним, что:
\begin{definition}
Приведённая схема --- схема, в которой все элементы выполняют разные функции, то есть не существует таких двух одинаковых элементов, на входы которых подаются одни и те же переменные или результаты вычисления других функций.
\end{definition}
%\textit{Заметим, что в данном определении под разными функциями понимаются неравные функции, либо равные функции, действующие на разном наборе переменных.}\\
Все булевы функции в этой лекции будем считать от переменных $\XXX$. Также все выкладки проводятся в выбранном <<стандартном>> базисе $\{\& ,\vee ,\neg\}$, если не указано обратное.
\subsection{Определения.}
\begin{definition}
$N_{=(n,l)}$ --- число приведённых схем сложности $l$ со входами $\XXX$.
\end{definition}
\begin{definition}
$N_{\leq(n,l)}$ --- число приведённых схем сложности не выше $l$ со входами $\XXX$.
\end{definition}
\subsection{Основная оценка.}
\begin{lemma}
При достаточно больших $n$ при $l\geq n \:\: \exists \, C>0$ выполняется неравенство\\
$$N_{\leq(n,l)}\leq (C\cdot l)^l $$
\end{lemma}
Пусть $S$ --- приведённая схема сложности $l$ со входами $\XXX$.
Пронумеруем элементы схемы и зафиксируем нумерацию $Num$. Пусть $L_i$ --- элемент схемы, имеющий в данной нумерации номер $i$. На множестве пар из схемы и нумерации на ней введём функцию
$t(S,Num)= T$%(аргумент функции --- пара $(S,Num)$, а значение --- \textit{таблица} $T$)
, где $T$ --- таблица вида:
$$
    \begin{array}{|c|c|c|}
    \hline
       f & E_1 & E_2 \\
      \hline
       \& & x_1 & x_3 \\
      \hline
       \vee & x_1 & x_3 \\
       \hline
       \neg & x_2 & x_2 \\
       \hline
     \vee & x_2 & L_2 \\
     \hline
      \& & L_6 & L_4\\
    \hline
    \vee & L_1 & L_3\\
    \hline
    \end{array}
    $$
Таблица состоит из $l$ строк и трёх столбцов, в строчке с номером $i$ в первом столбце стоит знак функции, которую реализует элемент $L_i$, а в двух других --- элементы множества $\{ L_1,\ldots,L_l, \XXX\} $, которые поступают на вход этой функции. %Это могут быть переменные или <<выходы>>  других элементов. %Иначе, $e_1,\,e_2\, \in \{ L_1,\ldots,L_l, \XXX\} $. 
Если функция в левом столбце --- отрицание, в два правых столбца запишем один и тот же элемент из вышеуказанного множества, над которым производится отрицание. Например, на схеме, задаваемой таблицей выше, переменные $x_1$ и $x_3$ передаются на элемент <<и>> (первая строчка), результат передаётся вместе с отрицанием переменной $x_2$ на вход элемента <<или>> (шестая строчка) и т.д. Также обозначим за $a$ номер строки с элементом, выход которого является выходом всей схемы (тут $a=5$). \\
По такой таблице, построенной по схеме и нумерации, можно однозначно восстановить схему $S$.\\
%Оценим число таблиц, соответствующих всем парам $(S,Num)$ заданной сложности $l$. Обозначим его $N_{l}$.\\
Обозначим за $N_l$ число таблиц, соответствующих всем парам $(S,Num)$ заданной сложности $l$. Имеет место оценка
$$ N_l\leq 3^l\cdot (l+n)^{2l} \cdot l \leq 3^l \cdot 2l^{2l}\cdot l = (3\cdot 2^2)^l\cdot l^{2l}\cdot l = 12^l \cdot l^{2l} \cdot l \leq 13^l\cdot l^{2l}.$$

Первое неравенство очевидно. Второе неравенство следует из предположения $n\leq l$, при котором мы доказываем лемму. Последнее неравенство верно асимптотически и следует из неравенства $12^l\cdot l\leq 13^l$.
\begin{statement}
Пусть схема $S$ --- приведённая, $Num_1 \neq Num_2$ --- две её нумерации, $t(S, Num_1)= T_1$;  $t(S, Num_2)= T_2$, тогда $T_1 \neq T_2$
\end{statement}
\begin{proof}
%Здесь остановили редактирование вечером 26го марта.
Предположим, что $T_1 = T_2$.

Введём на $S$ ещё одну монотонную нумерацию $Num_0$ и зафиксируем её. Дальше, перебирая по порядку нумерации $Num_0$ элементы схемы $S$ найдём первый элемент $L_i$ (по нумерации $Num_0$), такой, что он имеет в $Num_1$ и $Num_2$ номера $k_1$ и $k_2$, причём $k_1 \neq k_2$. Такой элемент существует, потому что $Num_1 \neq Num_2$. Рассмотрим строки $k_1$ и $k_2$ таблиц $T_1$ и $T_2$ соответственно. 

В первой их клетке стоит один и тот же знак, так как функция, которую реализует элемент, не зависит от нумерации. Для двух других клеток есть две возможности: либо там стоит знак переменной, тогда они тоже одинаковы, либо элемент множества $\{ L_1 \cdots L_l\} $, для каждой таблицы в своей нумерации. \\
Посмотрим, <<выход>> каких элементов может подаваться на <<вход>> элемента $L_i$ (в нумерации $Num_0$). Так как $Num_0$ --- монотонная, то  это могут быть только элементы с меньшим номером в данной операции.  Но для элементов с меньшим номером в $Num_0$ их номера в $Num_1$ и $Num_2$ совпадают. \\
Это значит, что строчки $k_1$ в $T_1$ и $k_2$ в $T_2$ одинаковые. Так как таблицы (по нашему предположению) одинаковые, то в таблице $T_1$ строчка с номером $k_2 \neq k_1$ совпадает со строчкой с тем же номером в $T_2$, которая, в свою очередь, совпадает со строчкой $k_1$ в $T_1$. Это значит, что в $T_1$ есть две одинаковые строчки. Другими словами, в схеме $S$ есть два элемента, реализующие одинаковые функции. Это противоречит с приведённостью $S$.\\
Итак, $T_1 \neq T_2$
\end{proof}
Значит, число таблиц, соответствующей какой-либо схеме равно числу способов пронумеровать элементы этой схемы.
Тогда, учитывая, что число способов пронумеровать $l$ элементов --- $l!$, и что $l! \geq (\frac{l}{3})^l $:\\
$$ N_{=(n,l)}=\frac{N_l}{l!} \leq \frac{13^l\cdot l^{2l}}{l!} \leq 39^l\cdot l^l $$\\
Тогда:
$$N_{\leq(n,l)} \leq \sum _{i=0} ^{l} N_{=(n,i)} \leq (l+1)\cdot 39^l \cdot l^l \leq (40l)^l$$
Данная оценка завершает доказательство леммы.
\begin{statement}
$$L(n) \geq \frac{2^n}{n}$$
\end{statement}
\begin{proof}
Положим $l_\varepsilon = (1-\varepsilon)\cdot \frac{2^n}{n}$, $0 < \varepsilon < 1$\\
При любом $\varepsilon$ из заданного интервала верна оценка:
$$ \log_2{\frac{N_{\leq(n,l_\varepsilon)}}{2^{2^n}} } \leq l_\varepsilon \cdot \log_2 {(C\cdot l_\varepsilon)} - 2^n \leq (1-\varepsilon)\cdot \frac{2^n}{n} \cdot \log_2 {2^n} - 2^n = -\varepsilon \cdot \frac{2^n}{n}$$
При $n \rightarrow +\infty$ отношение $\cfrac{N_{\leq(n,l_\varepsilon)}}{2^{2^n}} \;$ стремится к нулю.\\

Это значит, что при достаточно больших $n$ число функций, которые можно реализовать при помощи схем, сложности меньше $\frac{2^n}{n}\,$, много меньше числа всех функций от $n$ переменных. А это значит, что существуют функции, сложность которых больше или равна $\frac{2^n}{n}\,$. А это и значит, что 
$$L(n) \geq \frac{2^n}{n}$$
\end{proof}
\begin{theorem}
Пусть $n \rightarrow + \infty \; $, тогда 
$$L(n) - \frac{2^n}{n} \geq \frac{2^n \log_2 n}{n^2} $$
\end{theorem}
\begin{proof}
Пусть $\varepsilon > 0 \;$ --- фиксировано. Положим $l_\varepsilon = \frac{2^n}{n} + (1-\varepsilon) \cdot \frac{2^n \log_2 n}{n^2}\,$.\\
Так как $\log_2 n \leq n $ при больших $n$, то $l_\varepsilon = \frac{2^n}{n} + (1-\varepsilon) \cdot \frac{2^n \log_2 n}{n^2} \leq 2\cdot \frac{2^n}{n}$ и $C\cdot l_\varepsilon \leq 2C\cdot \frac{2^n}{n}$.
%$$ \log_2{\frac{N_{\leq(n,l_\varepsilon)}}{2^{2^n}}} \leq l_\varepsilon \cdot \log_2{C\cdot l_\varepsilon} - 2^n \leq l_\varepsilon \cdot \log_2{2C\cdot \frac{2^n}{n}} - 2^n \leq (\frac{2^n}{n} + (1-\varepsilon)\frac{2^n \cdot \log_2 n}{n^2})\cdot (\log_2 2C + n - \log_2 n) - 2^n = $$

$$ \log_2{\frac{N_{\leq(n,l_\varepsilon)}}{2^{2^n}}} \leq \log_2 {\frac{(Cl_\varepsilon)^{l_\varepsilon}}{2^{2^n}}} = l_\varepsilon \log_2 Cl_\varepsilon - 2^n = (*).$$
Подставляем выражение для $l_\varepsilon$ перед логарифмом в $(*)$, получаем:
$$(*) = \Big(\frac{2^n}{n} + (1-\varepsilon)\frac{2^n\log_2 n}{n^2}\Big)\cdot \log_2(Cl_\varepsilon) - 2^n 
\leq \Big(\frac{2^n}{n} + (1-\varepsilon)\frac{2^n\log_2 n}{n^2}\Big)\cdot\log_2(2C\cdot \frac{2^n}{n}) - 2^n =$$ 
$$ = \Big( \frac{2^n}{n} + \frac{2^n\log_2 n}{n^2} - \varepsilon\frac{2^n\log_2 n}{n^2} \Big)\cdot (\log_2 2c + n - \log_2 n ) - 2^n = $$ $$= 2^n - \frac{2^n\log_2n}{n} + \frac{2^n\log_2n}{n} - \varepsilon\frac{2^n\log_2n}{n} - 2^n + \bar{o}\big( \frac{2^n\log_2n}{n} \big) = - \varepsilon\frac{2^n\log_2n}{n} + \bar{o}\big( \frac{2^n\log_2n}{n}\big).$$

Получили оценку 
$\log_2{\frac{N_{\leq(n,l_\varepsilon)}}{2^{2^n}}} \leq - \varepsilon\frac{2^n\log_2n}{n} + \bar{o}\big( \frac{2^n\log_2n}{n}\big)$; правая часть стремится к $-\infty$ при $n \rightarrow +\infty$, значит, $\frac{N_{\leq(n,l_\varepsilon)}}{2^{2^n}} \rightarrow 0$, значит, $L(n) \geq l_\varepsilon$. 

Рассуждение верно для любого сколь угодно малого $\varepsilon > 0$. Переходя к пределу при $\varepsilon \rightarrow 0$, получаем утверждение теоремы.
\end{proof}



\end{document}
