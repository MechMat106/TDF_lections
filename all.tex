\documentclass[12pt]{article}
\usepackage[utf8]{inputenc}
\usepackage[russian]{babel}
\usepackage[dvips]{graphicx}
\usepackage{pscyr}
\usepackage[T1]{fontenc}
\usepackage{amssymb, amsmath, textcomp, amsthm, multicol}
\usepackage{scalerel}
\textwidth=500pt
\textheight=750pt
\oddsidemargin=20pt
\hoffset=-1.5cm
\topmargin=-25mm

%\usepackage{xcolor}
\usepackage{hyperref}
%\definecolor{linkcolor}{HTML}{000000}
%\definecolor{urlcolor}{HTML}{000000}
\hypersetup{pdfstartview=FitH, linkcolor=linkcolor, urlcolor=urlcolor, colorlinks=true}

\newtheoremstyle{neosn}{0.5\topsep}{0.5\topsep}{\rm}{}{\sc}{.}{ }{{\bf \thmname{#1}}\thmnote{ {\mdseries#3}}}
\newtheorem{theorem}{Теорема}
\newtheorem{lemma}{Лемма}
\theoremstyle{neosn}
\newtheorem{proposition}{Предложение}
\newtheorem{statement}{Утверждение}
\newtheorem{corollary}{Следствие}
\newtheorem{definition}{Определение}
\newtheorem{example}{Пример}
\renewcommand{\proofname}{Доказательство}
\newcommand{\XXX}{x_1, \ldots, x_n}
\newcommand{\SIG}{\sigma_1, \ldots, \sigma_n}

%команда для амперсанда, ага.
\DeclareMathOperator*{\amper}{\scalerel*{\&}{\sum}}


\begin{document}
\section{Булевы функции}
\subsection{Определение булевой функции.}
Обозначим за $E$ множество $\lbrace0,1\rbrace$.

\begin{df}
$f(\XXX)\in{E}$ --- функция алгебры логики {\bf (булева функция)}, где $x_{i}\in{E} ~ \forall i=1,\ldots,n$ --- это отображение $f\colon E^{n}\rightarrow{E}$. Его можно проиллюстрировать таблицей возможных значений $f$ на различных наборах переменных:\\

$$\begin{array}{|ccccc|c|}
\hline
x_1 & x_2 & \ldots & x_{n-1} & x_n & f(\XXX)  \\
0 & 0 &\ldots & 0 & 1 & 0 ~ \textbf{или} ~ 1 \\
0 & 0 &\ldots & 1 & 1 & 0 ~ \textbf{или} ~ 1 \\
\ldots & \ldots & \ldots & \ldots & \ldots & \ldots\\
1 & 1 &\ldots & 1 & 1 & 0 ~ \textbf{или} ~ 1 \\
\hline
\end{array}$$
\\
\\\\
\end{df}

\begin{definition}
$P_{2}$ --- множество всех булевых функций от произвольного конечного множества переменных. $P_2(n)$ --- множество всех булевых функций от $n$ переменных. 
\end{definition}
\begin{definition}
$E^{n}=\{(\SIG)|~ \sigma_{i}\in E; ~ i=1,\ldots,n\}$ \\
\end{definition}
 
\begin{statement}
 $|P_2(n)|=2^{2^{n}}$. \\
\end{statement}
\begin{proof}
Очевидно.
\end{proof}
\subsection{Существенные и фиктивные переменные.} 
\begin{definition}
	Пусть $f(\XXX)$ --- булева функция. 
	Тогда $x_i$ называется \textbf{существенной} переменной для $f$, если:  $\exists{\sigma_1,\sigma_2, \ldots \sigma_{i-1}, \sigma_{i+1}, \ldots, \sigma_{n}}\in\{0,1\}$, такие, что: 

$ f(\sigma_1,\sigma_2, \ldots ,\sigma_{i-1}, 0, \sigma_{i+1}, \ldots, \sigma_{n})\neq f(\sigma_1,\sigma_2, \ldots, \sigma_{i-1}, 1, \sigma_{i+1}, \ldots, \sigma_{n}).$
	В противном случае переменная называется \textbf{фиктивной} (пример придумать не очень сложно).
\end{definition}
%\begin{enumerate}

1. Пусть $x_i$ --- фиктивная переменная для $f$. Рассмотрим функцию $g(x_1,x_2,\ldots,x_{i-1},x_{i+1}, \ldots, x_{n}),\\
    g(\sigma_1,\sigma_2 \ldots \sigma_{i-1}, \sigma_{i+1}, \ldots, \sigma_{n})=f(\sigma_1,\sigma_2 \ldots \sigma_{i-1},0, \sigma_{i+1}, \ldots, \sigma_{n}) =f(\sigma_1,\sigma_2 \ldots \sigma_{i-1},1, \sigma_{i+1}, \ldots, \sigma_{n}). $ Тогда говорят, что  \textbf{$g$ получена из $f$ удалением фиктивной переменной $x_i$}.

2. Пусть $f(\XXX)$ --- булева функция. Также, пусть имеется $y \neq \XXX $. Рассмотрим функцию $h(\XXX,y)$, $h(\SIG,\sigma)=f(\SIG)$. Тогда говорим, что \textbf{$h$ получена из $f$ добавлением фиктивной переменной  $y$.}

%\end{enumerate}
\begin{definition}
	Две булевы функции называются \textbf{равными}, если они могут быть получены друг из друга с помощью некоторого числа операций добавления или удаления фиктивных переменных. \\
\end{definition}
\subsection{Элементарные функции:} 
\begin{enumerate}
	\item От одной переменной.
	$$
    \begin{array}{|c|c|c|c|c|}
    \hline
    x & 0 & x & \bar{x} & 1 \\
    \hline
    0 & 0 & 0 & 1 & 1 \\
    \hline
    1 & 0 & 1 & 0 & 1 \\
    \hline
    \end{array}
    $$
    \item От двух переменных:
    $$
    \begin{array}{|c|c|c|c|c|c|c|c|c|c|}
	\hline
	 x & y & xy & x\vee y & x\oplus y & x\sim y & x\rightarrow y & x|y & x\downarrow y\\
	\hline
	 0 & 0 & 0 & 0 & 0 & 1 &  1 & 1 & 1 \\
	\hline
	 0 & 1 & 0 & 1 & 1 & 0 &  1 & 1 & 0 \\
	\hline
	 1 & 0 & 0 & 1 & 1 & 0 &  0 & 1 & 0  \\
	\hline
	 1 & 1 & 1 & 1 & 0 & 1 &  1 & 0 & 0 \\
	\hline
	\end{array}
	$$
	\item
	От трех переменных (функция "медиана"):
	$$
	\begin{array}{rrr|c}
	x~~ & y~~ & z~~ & f(x,y,z)\\
	\hline
	\begin{array}{r} % Вложенная таблица для каждого столбца
	0\\ 0\\ 0\\ 0\\ 1\\ 1\\ 1\\ 1\\
	\end{array}
	&
	\begin{array}{r}
	0\\ 0\\ 1\\ 1\\ 0\\ 0\\ 1\\ 1\\
	\end{array}
	&
	\begin{array}{r}
	0\\ 1\\ 0\\ 1\\ 0\\ 1\\ 0\\ 1\\
	\end{array}
	&
	\begin{array}{r}
	0\\ 0\\ 0\\ 1\\ 0\\ 1\\ 1\\ 1\\
	\end{array}
	\end{array}
	$$
\end{enumerate}	


\subsection{Формула над системой булевых функций.}
$\Phi=\{f_{1}(x_{1},x_{2},...,x_{n_1});f_{2}(x_{1},x_{2},...,x_{n_2});...;f_{n}(x_{1},x_{2},...,x_{n_n})\}\subseteq P_2$ --- некоторое множество булевых функций, таких что каждой булевой функции $f_{i}(x_{1},x_{2},...,x_{n_i})$ сопоставляем функциональный символ $f_{i}$.
\begin{definition}

	\textit{Формулой над $\Phi$} называется строка символов, состоящая из любых символов-переменных, обозначающих $f_1,...,f_n$ и вспомогательных символов $"("$,$")"$ ,$","$, определяемое индуктивным образом: 

\textbf{База индукции:} символ любой переменной --- правильная формула над $\Phi$.

\textbf{Индуктивное предположение: } пусть $F_1,F_2,...,F_{n_i}$ --- некоторые формулы над $\Phi$, тогда $f_i(F_1,F_2,...,F_{n_i})$ --- тоже формула над $\Phi$.

\end{definition}
\begin{example}
	$((\overline{x\vee y}) \& (z\rightarrow y ))$ --- формула над $\{x \vee y; x \& y, x \rightarrow y, \overline{x} \}$\\
\end{example}
Конъюнкция имеет приоритет над дизъюнкцией.\\

%\newpage 
%зачем тут newpage?
Значения формулы на наборе значений переменных, входящих в формулу, определяется индуктивным образом.

\textbf{База индукции:} если $f$ --- тривиальная, то все очевидно.

\textbf{Индуктивное предположение:} пусть $F_1,F_2, \ldots, F_n$ --- формулы, для которых данное понятие уже определено. \\
$F=f_i(F_1,F_2, \ldots, {F_{n_i}});$ \\
$\XXX$ --- все переменные, содержащиеся в F. \\
$\Omega=(\SIG)$ --- набор значений $\XXX$. \\
$\Omega_j$ --- поднабор значений из $\Omega$ для переменных, содержащихся в формуле $F_j$. \\
$b_j$ --- значение функции $F_j$ на наборе $\Omega_j$. \\
Тогда значение $F$ на наборе $\Omega$ равно $f_i(b_1, \ldots, {b_n}_i)$ \\
Пусть $F$ --- формула над $\Phi$, содержащая символы переменных $x_1, \ldots, x_n$. Тогда $F$ реализует функцию $f(\XXX)$, т.ч для любого набора $(\sigma_1, \ldots, \sigma_n)$ значений $x_1,...,x_n$ значение $f(\sigma_1, \ldots, \sigma_n)$ равно значению формулы $F$ на $\sigma_1, \ldots, \sigma_n$. \\
$f$ получается из $\Phi$ с помощью операции суперпозиции, если F реализуется некоторой нетривиальной формулой над $\Phi$.

\begin{definition}
Две формулы $F_1$ и $F_2$ называются {\bf эквивалентными}, если они реализуют одинаковые функции.
\end{definition}

Пусть $\ast \in \{\vee, \&, \oplus, \sim \}$ --- некоторая операция.

\begin{enumerate}
	\item 
		$x \ast y = y \ast x$ (коммутативность)
	\item 
		$x \ast (y \ast z) = (x \ast y) \ast z$ (ассоциативность)
	\item 
		$x (y \vee z) = xy \vee xz$

		$x (y \oplus z) = xy \oplus xz$

		$x \vee (y  \&  z) = (x \vee y) \& (x \vee z)$

		$x \vee (y  \sim  z) = (x \vee y) \sim (x \vee z)$ (дистрибутивность)
	\item
		$x \vee xy = x$ (поглощение)
	\item 
		$\overline{\overline{x}} = x$ (двойное отрицание)
	\item 
		$\overline{x \vee y} = \overline{x} \& \overline{y}$

		$\overline{x \& y} = \overline{x} \vee \overline{y}$ (закон де Моргана)
	\item 
		$x\overline{x} = 0$, \smallskip $x \vee \overline{x} = 1$, \smallskip $x \oplus \overline{x} = 1$, \smallskip $x \sim \overline{x} = 0$

		$xx = x$, \smallskip $x \vee x = x$, \smallskip $x \oplus x = 0$, \smallskip $x \sim x = 1$

		$x \& 1 = x$, \smallskip $x \vee 1 = 1$, \smallskip $x \oplus 1 = \overline{x}$, \smallskip $x \sim 1 = x$

		$x \& 0 = 0$, \smallskip $x \vee 0 = x$, \smallskip $x \oplus 0 = x$, \smallskip $x \sim 0 = \overline{x}$
\end{enumerate}

\section{Лекция 2 (Замыкания и прочее).}

\subsection{Определения.}

Возьмем множество $F \subseteq P_2$.

\begin{definition}
	Замыкание $[F]$ множества $F$ --- это множество всех булевых функций, получаемых из булевых функций множества $F$ с помощью операций суперпозиции, удаления и добавления фиктивных переменных.
\end{definition}

\begin{definition}
	$F$ --- замкнуто, если $[F] = F$.
\end{definition}

\begin{enumerate}
	\item
	$[\{x \oplus y\}] = \{0, x, x_1 \oplus \ldots \oplus x_t (t \ge 2)\}$
	\item
	$P_2$ --- замкнуто.	
\end{enumerate}

\begin{definition}
	$P_2(n)$ --- все булевы функции, существенно зависящие от не более, чем $n$ переменных.
\end{definition}

\begin{enumerate}
	\item
	$P_2(1)$ --- замкнуто.
	\item
	$P_2(2)$ --- не замкнуто. $\left( xy \in P_2(2), xyz \not\in P_2(2) \right)$
\end{enumerate}
\subsection{Свойства замыкания.}
\begin{enumerate}
	\item $F \subseteq [F].$
	\item $F_1 \subseteq F_2 \Longrightarrow [F_1] \subseteq [F_2]$
	\item $[[F]] = [F]$
	\begin{proof}
		1) $[F] \subseteq [[F]]$ (по 1, 2)

		2)$[[F]] \subseteq [F]$.

		$f(\XXX) \in [[F]] \Rightarrow \exists$ формула $\Phi$, реализующая $f$. Пусть $f_1, \ldots, f_s$ --- все функциональные символы, содержащиеся в $\Phi$. $f_1, \ldots, f_s \in [F] \Rightarrow $ каждая функция $f_i$ реализуется некоторой формулой $\Phi_i$ над $F$ : $\Phi = f_i(F_1, \ldots, F_{n_i})$.

		$\Phi_i(F_1, \ldots, F_{n_i})$ --- формула, полученная из $\Phi$ заменой $x_i \longmapsto F_i$. $\Phi_i(F_1, \ldots, F_n).$

		$\Phi_i(F_1, \ldots, F_n).$

		Так получим: 

		$\Phi'$ --- формулу над $F$, реализующую функцию $F \Rightarrow f \in [F] \Rightarrow [[F]] \subseteq [F]$.
	\end{proof}
	\item  $[F_1] \cap [F_2]$ -- замкнуто.
	\begin{proof}
		Возьмем $f \in [[F_1] \cap [F_2]]$: $f$ реализуется формулой $\Phi$ над $[F_1] \cap [F_2]$. Пусть $f_1, \ldots f_s$ -- все функциональные символы из $\Phi$. $\forall \: i \: f_i$ реализуется и формулой $\Phi_1$ над $F_1$ и формулой $\Phi_2$ над $F_2 \Rightarrow f \in  [F_1] \cap [F_2]$.
	\end{proof}
	\item  $[F_1] \cup [F_2]$ не обязательно замкнуто.
\end{enumerate}

\subsection{}
Пусть $F$ --- замкнутое множество, и $F_1 \subseteq F$.

\begin{definition}
	$F_1$ называется полным в $F$, если $[F_1] = F$.
\end{definition}

\begin{definition}
	$F_1$ называется полным, если $[F_1] = P_2$.
\end{definition}

\begin{example}
	$P_2$ -- полное множество.
\end{example}

\begin{statement}
	$f(\XXX)$ -- булева функция. Тогда: $f(\XXX) = (\overline{x_1} \: \& \: f(0, x_2, \ldots, x_n)) \vee (x_1 \: \& \: f(1, x_2, \ldots , x_n))$
\end{statement}
\begin{proof}
	Пусть $\sigma = (\SIG) $ --- набор значений $\XXX$.
\begin{enumerate}
\item $\sigma_1 = 0$. 

$\overline{\sigma_1} \: \& \: f(0, \sigma_2, \ldots, \sigma_n) \vee \sigma_1 \: \& \: f(1, \sigma_2, \ldots, \sigma_n) = 1 \: \& \: f(0, \sigma_2, \ldots, \sigma_n) \vee 0 \: \& \: f(1, \sigma_2, \ldots, \sigma_n) = f(0, \sigma_2, \ldots, \sigma_n) = f(\SIG) $

\item $\sigma_1 = 1$.

$0 \: \& \: f(0, \sigma_2, \ldots, \sigma_n) \vee 1 \: \& \: f(1, \sigma_2, \ldots, \sigma_n) = f(1, \sigma_2, \ldots, \sigma_n) = f(\SIG) $
\end{enumerate}
\end{proof}

$
f(\XXX) = (\overline{x_1} \: \& \: f(0, x_2, \ldots, x_n)) \vee (x_1 \: \& \: f(1, x_2, \ldots , x_n)) = $

$= \overline{x_1} \: \& \: (\overline{x_2} \: \& \: f(0, 0, \ldots, x_n)) \vee (x_2 \: \& \: f(0, 1, \ldots , x_n)) \vee (x_1 \: \& \: (\overline{x_2} \: \& \: f(1, 0, \ldots, x_n)) \vee (x_1 \: \& \: f(1, 1, \ldots , x_n))) =$

$
\overline{x_1}\overline{x_2}f(0, 0, \ldots, x_n) \vee \overline{x_1} x_2 f(0, 1, \ldots, x_n) \vee x_1 \overline{x_2}f(1, 0, \ldots, x_n) \vee x_1 x_2 f(1, 1, \ldots, x_n)
$

\begin{definition}
	$x_\sigma = \begin{cases} x \text{, если }\sigma = 1 \\ \overline{x} \text{, если }\sigma = 0 \end{cases}$
\end{definition}


Итак, $f(\XXX)$ можно переписать в виде $\bigvee \limits_{\sigma_1, \sigma_2 \in E} f(\sigma_1, \sigma_2, x_3, \ldots, x_n)$.

Мы также можем разложить $f$ по $k$ переменным:

$f(\XXX) = \bigvee \limits_{(\sigma_1,\ldots, \sigma_k) \in E^k} f(\sigma_1, \ldots, \sigma_k, \ldots, x_n)$

При $k = n$ получаем:  $f(\XXX) = \bigvee \limits_{(\sigma_1,\ldots, \sigma_n) \in E^n} x_1^{\sigma_1}\ldots x_n^{\sigma_n}f(\SIG) =$

$ = \bigvee \limits_{\substack{(\sigma_1,\ldots, \sigma_n) \in E^n \\ f(\SIG) = 1}} x_1^{\sigma_1}\ldots x_n^{\sigma_n} $ -- {\large Совершенная дизъюнктивная нормальная форма ({\bf СДНФ}).}

\begin{statement}
$	\{x\&y, x \vee y, \bar{x} \} - \text{полное множество.} $
\end{statement}
\begin{proof}
    Если $f \ne 0$, то СДНФ - формула над $\{x\&y, x \vee y, \bar{x} \}$
    Если $f = 0$, то $f = \overline{x} \& x \Rightarrow $ любая функция реализуется формулой над $\{x\&y, x \vee y, \bar{x} \}$.
\end{proof}

\begin{lemma}[О сводимости полных множеств.]
	$F, F' \subseteq P_2$, $F$ -- полное множество и любая функция из $F$ может быть реализована формулой над $F' \Rightarrow F'$ -- полное множество.
\end{lemma}
\begin{proof}
	$\forall$ функция из $F$ может быть реализована формулой над $F' \Rightarrow F \subseteq [F'] \Rightarrow [F] \subseteq [[F']] = [F'].$

	$F$ -- полное $\Rightarrow [F] = P_2, [F] \subseteq [F'] \Rightarrow P_2 \subseteq [F'] \Rightarrow F'$ -- полное.
\end{proof}

\newpage

\begin{statement}
$	\{x\&y, \bar{x} \} - \text{полное множество.} $
\end{statement}
\begin{proof}
			$\{x\vee y, x \& y, \bar{x} \}  - \text{полное множество. Учитывая, что: } 
			x \vee y = \overline{\bar{x}\&\bar{y} } $ , то по лемме о сходимости получаем нужное. 

\end{proof}	
\begin{statement}
	$\{x \vee y, \bar{x} \}$ -- полное множество.  
\end{statement}
\begin{proof}
	$	\{x\&y, \bar{x} \} - \text{полное множество.} $ Учитывая, что: $x \& y =  \overline{\bar{x}\vee\bar{y} }$, то по лемме о сходимости получаем нужное.
\end{proof}
\newpage
\begin{statement} 
$\{x \oplus y; x \& y, 1\}$ -- полное множество.
\end{statement}
\begin{proof}
	$\bar{x}=x \oplus 1. $ Получаем нужное по лемме о сходимости и утверждению 2.
\end{proof}	
\begin{statement}
	$\{x | y\}$ -- полное множество. 
\end{statement}
\begin{proof}
	$x|y=\bar{x} \vee \bar{y}=\overline{x \& y}$ \\
	$\bar{x}=x | x. \\
	x \& y=\overline{x|y}=(x|y)|(x|y) \\
	{x \& y, \bar{x}} \text{ -- полное по лемме о сходимости.} \Rightarrow {x|y}$ - полное.   
\end{proof}
\begin{corollary}
	Из любого полного множества можно выделить конечное полное подмножество.
\end{corollary}
\begin{proof}
	$F\subseteq P_2$ -- полное множество. $\Rightarrow$ существует формула Ф над $F$, реализующая ${x|y}$. Пусть $\{f_1, \ldots, f_s\}$ - множество всех символов функций, содержащихся в Ф.Ф -- формула над $\{f_1, \ldots, f_s\}$ $\Rightarrow  ~ x|y$ содержится в замыкании. $\{x|y\}$ -- полное. $\Rightarrow$ по лемме о сводимости $\{f_1, \ldots, f_s\} \subseteq F$ -полное.
\end{proof}  
\begin{document}

\begin{statement}
$	\{x\&y, \bar{x} \} - \text{полное множество.} $
\end{statement}
\begin{proof}
			$\{x\vee y, x \& y, \bar{x} \}  - \text{полное множество. Учитывая, что: } 
			x \vee y = \overline{\bar{x}\&\bar{y} } $ , то по лемме о сходимости получаем нужное. 

\end{proof}	
\begin{statement}
	$\{x \vee y, \bar{x} \}$ -- полное множество.  
\end{statement}
\begin{proof}
	$	\{x\&y, \bar{x} \} - \text{полное множество.} $ Учитывая, что: $x \& y =  \overline{\bar{x}\vee\bar{y} }$, то по лемме о сходимости получаем нужное.
\end{proof}
\begin{statement} 
$\{x \oplus y; x \& y, 1\}$ -- полное множество.
\end{statement}
\begin{proof}
	$\bar{x}=x \oplus 1. $ Получаем нужное по лемме о сходимости и утверждению 2.
\end{proof}	
\begin{statement}
	$\{x | y\}$ -- полное множество. 
\end{statement}
\begin{proof}
	$x|y=\bar{x} \vee \bar{y}=\overline{x \& y}$ \\
	$\bar{x}=x | x. \\
	x \& y=\overline{x|y}=(x|y)|(x|y) \\
	{x \& y, \bar{x}} \text{ -- полное по лемме о сходимости.} \Rightarrow {x|y}$ - полное.   
\end{proof}
\begin{corollary}
	Из любого полного множетсва можно выделить конечное полное подмножество.
\end{corollary}
\begin{proof}
	$F\subseteq P_2$ -- полное множество. $\Rightarrow$ существует формула Ф над $F$, реализующая ${x|y}$. Пусть $\{f_1, \ldots, f_s\}$ - множество всех символов функций, содержащихся в Ф.Ф -- формула над $\{f_1, \ldots, f_s\}$ $\Rightarrow  ~ x|y$ содержится в замыкании. $\{x|y\}$ -- полное. $\Rightarrow$ по лемме о сводимости $\{f_1, \ldots, f_s\} \subseteq F$ -полное.
\end{proof}  
\subsection{Полином Жегалкина.} 
$f(x_1, \ldots, x_n)$ -- булева функция. $f\neq 0$. \\
$f(x_1, \ldots, x_n)=\bigvee \limits_{f(\sigma_1, \ldots, \sigma_n)=1} x^{\sigma_1}, \ldots, x^{\sigma_n}=\bigoplus \limits_{f(\sigma_1, \ldots, \sigma_n)=1} x^{\sigma_1}, \ldots, x^{\sigma_n}=$ 
\begin{flushright}
$
=\bigoplus \limits_{f(\sigma_1, \ldots, \sigma_n)=1} (x_1\oplus \bar{\sigma_1}),\ldots,(x_n\oplus \bar{\sigma_n})=\bigoplus \limits_{i_1<i_2< \ldots,<i_k} c_{i_1, \ldots, i_k}x_{i_1}, \ldots ,x_{i_k} \bigoplus C $, где  $c_{i_1, \ldots, i_k}=\{0,1\}$.
\end{flushright} 
$0$ - полином Жегалкина для $f=0$.(по определению) 

\begin{statement}
	Любая булева функция может быть реализована единственным полиномом Жегалкина.(с точностью до перестановки слагаемых и сомножителей). 
\end{statement}
\begin{proof}
	Полином Жегалкина единственен, так как $c_{i_1}, \ldots, c_{i_n}$ -- однозначно определены $2^n$ -  коэффицентов $2^{2^n}$ - набор значений коэффицентов $2^{2^n}$ - количество функций от $n$ - переменных $\Rightarrow$ для каждой функции $\exists !$ полином Жегалкина. 
\end{proof}
\subsection{Функции, сохраняющие ноль и единицу. \\}
\begin{definition}
	$f$ - сохраняет 0, если $f(0,\ldots, 0)=0$.
	$T_0$ - множество всех функций, сохраняющих ноль.(например, $0,x,x \& y, x \vee y, x \oplus y)$.
\end{definition}
\begin{definition}
	Селекторная функция - функция, тождественна равная переменной. \\
\end{definition}
\begin{lemma}
	$T_0 $ - замкнуто. 
\end{lemma}
\begin{proof}
	Тождественная функция содержится в $T_0$.Значит, надо проверить, что если  \\
	 $g(x_1,\ldots, x_n),g_1, \ldots, g_n \in T_0, \text{,то~} f(g_1, \ldots, g_n) \in T_0. \\$
	Можем полагать, что $g_1, \ldots, g_n$ зависят от одних и тех же переменных: $x_1, \ldots, x_n$(иначе можно добавить переменные в качестве фиктивных).Тогда:\\
	$f(g_1(x_1, \ldots, x_n),\ldots, g_n(x_1, \ldots, x_n))=h(x_1, \ldots, x_n) \\
	h(0, \ldots, 0)=f(g_1(0,\ldots,0),\ldots,g_n(0, \ldots, 0))=f(0,\ldots,0)=0. \Rightarrow h \in T_0. 
$
\end{proof}
\begin{definition}
	$f$ - сохраняет 1, если $f(1,\ldots, 1)=1$.
	$T_1$ - множество всех функций, сохраняющих ноль.(например, $(1,x,xy,x\rightarrow y, x \vee y)$
\end{definition}
\begin{lemma}
	$T_1 $ является замкнутым классом. 
\end{lemma}
\begin{proof}
	Аналогично предыдущей лемме. 
\end{proof}
\subsection{Монотонные функции.}	
Определим правило сравнения на наборах из нулей и единиц.\\
$\sigma'=\{\sigma_1',\ldots,\sigma_n'\}, \sigma''= \{\sigma_1'',\ldots,\sigma_n''\} \in \{0,1\}^n.$ \\
Будем говорить, что $\sigma'<\sigma'', \text{если} ~ \sigma_i ' < \sigma_i '' ~ \forall i.$ \\
Заметим, что существуют несравнимые наборы, например: (101) и (010). 
\begin{definition}
$f$ --монотонная, если для любых $\sigma' \text{ и } \sigma'' ~ \text{таких, что:} ~ \sigma'\leq \sigma'' $выполняется, что:$ f(\sigma')\leq f(\sigma''). $
\end{definition}
\begin{lemma}
	M является замкнутым классом. 
\end{lemma}
\begin{proof}
	%Воткнуть в обозначения лекций я так и не смог, потому частично взял с Лупанова. 
Тождественная функция содержится в M. Значит, осталось проверить, что если
$f(x_1, \ldots,x_k), g_1, \ldots, g_k \in M$, то $h=f(g_1,\ldots,g_n) \in M $. Можно считать, что  $g_1,\ldots g_n$  -- функции от одного и того же количества переменных, в противном случае недостающие переменные можно добавить в качестве несущественных.  Выберем произвольные различные наборы $ 
\sigma'=\{\sigma_1',\ldots,\sigma_n'\}, \sigma''= \{\sigma_1'',\ldots,\sigma_n''\}$, такие что: $ \sigma' \leq \sigma''.\\
$Рассмотим $ h(\sigma')=f(g_1(\sigma'),g_2(\sigma'), \ldots, g_k(\sigma')) ~\text{и}~ h(\sigma'')=f(g_1(\sigma''),\ldots,g_k(\sigma'')). \\
g_i(\sigma')<g_i(\sigma''), $ так как $ g_i$ - монотонная. $ f(g_1(\sigma'),g_2(\sigma'), \ldots, g_k(\sigma')) \leq f(g_1(\sigma''),g_2(\sigma''), \ldots, g_k(\sigma''))$, так как f- монотонная.$\Rightarrow h $ - тоже монотонная. 
\end{proof}
\begin{lemma}[О немонотонных функциях]
	$f(x_1,\ldots, x_n)\notin M. $ Тогда $\bar{x} \in [\{f;0;1\}].$
\end{lemma}	
\begin{proof}
	$f \notin M \Rightarrow \exists \sigma' \text{и} ~ \sigma''$,такие что $ \sigma \leq \sigma''~ \text{и}~ f(\sigma')=1, f(\sigma'')=0, \text{где }  \sigma',  \sigma'' \text{-разные}. \\
$Без ограничения общности будем считать, что $\sigma' ~\text{и} ~ \sigma''$ устроены следующим образом: \\$ 
	\sigma'=(0,\ldots,0,\ldots,0,1,\ldots,1) \\
	\sigma''=(\underbrace{1,\ldots,1}_k,\underbrace{0,\ldots,0}_s,\underbrace{1,\ldots,1}_{n-k-s}) \\
	g(x)=f(\underbrace{x, \ldots, x}_k, \underbrace{0, \ldots, 0}_s,\underbrace{1,\ldots,1}_{n-k-s}) =\bar{x}.$ Так как, ($g(0)=1,g(1)=0$).
\end{proof}
\subsection{Самодвойственные функции.}
	$f^*(x_1,\ldots,x_n)=\overline{f(\bar{x_1},\ldots,\bar{x_n})}.$ $f^*$ называется двойственной функцией к $f$.($(x\&y)^*=x \vee y$)	
	Легко заметить, что $(f^*)^*=f.$
	\begin{definition}
		Самодвойственная функция -- функция, двойственная сама себе, множество всех таких функций обозначается $S$.
	\end{definition}
\begin{statement}
	$\bar{x},x\oplus y \oplus z, m(x,y,z) \in S; ~
	0,1,x\oplus y, x\textrightarrow y, x \& y, x \vee y \notin S $ 
\end{statement}
\begin{proof}
	Явной проверкой в этом несложно убедиться. 
\end{proof}	
\begin{lemma}
	$S$ является замкнутым классом.
\end{lemma}
\begin{proof}
	Тождественная функция содержится в S. Значит, осталось проверить, что если $f(x_1, \ldots,x_k), g_1, \ldots, g_k \in M$, то $h=f(g_1,\ldots,g_k) \in M $. \\
	$h^*(x_1,\ldots,x_n)=\overline{f(g_1(\bar{x_1},\ldots,\bar{x_n})),\ldots,g_k(\bar{x_1},\ldots,\bar{x_n})}=\overline{f(\overline{\overline{g_1(\bar{x_1},\ldots,\bar{x_n})}},\ldots,\overline{\overline{g_k(\bar{x_1},\ldots,\bar{x_n})}}}; \\\overline{f(\overline{g^*_{1}({x_1},\ldots,{x_n})},\ldots,\overline{g^*_k({x_1},\ldots,{x_n})}} $ \\
	Так как $g_1,\ldots,g_k \in S \Rightarrow g_1=g^*_1;\ldots;g_k=g^*_k$, то: \\
	$h=\overline{f(\overline{g_{1}({x_1},\ldots,{x_n})},\ldots,\overline{g_k({x_1},\ldots,{x_n})}}=f^*(g_1(x_1,\ldots, x_n),\ldots,g_k(x_1,\ldots,x_n)).\\ $
	$f \in S \Rightarrow f^*=f$, значит \\
	$h^*(x_1,\ldots,x_n)=f^*(x_1,\ldots,x_n)=f(x_1,\ldots,x_n)=h(x_1,\ldots,x_n). \Rightarrow h \in S.$
\end{proof}
\end{document}

\end{document}