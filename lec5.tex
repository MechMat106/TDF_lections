\subsection{Предполные классы}

\begin{definition}
	Пусть $F \subseteq P_2$. Тогда $F$ -- предполный, если: 
	\begin{enumerate}
		\item $F \ne P_2$
		\item $[F] = F$
		\item $\forall f \in F: [F \cup {f}] = P_2$
	\end{enumerate}
\end{definition}

В клетке таблице снизу стоит функция $\in$ строке и $\not \in $ столбцу $\Rightarrow$ ни один из классов не содержится в другом.

$$\begin{array}{|c|c|c|c|c|c|}
\hline
 & T_0 & T_1 & L & S & M  \\
\hline
T_0 &  & 0 & xy & xy & x \oplus y \\
\hline
T_1 & 1 & &  xy & xy &  x ~ y \\
\hline
L & \overline{x} & \overline{x} &  & x \oplus y & \overline{x}\\
\hline
S & \overline{x} & \overline{x} & m(x, y, z) &  & \overline{x} \\
\hline
M & 1 & 0 & xy & xy &  \\
\hline
\end{array}$$
\\
\\\\

\begin{theorem}
	$T_0, T_1, S, M, L$ -- множеcтво всех предполных классов.
\end{theorem}

\begin{proof}
	Пусть есть $A = [A]$ -- предполный. Возьмем класс $M$.

	Пусть $A \subset M \Rightarrow \exists f \in M \setminus A : [A \cup {f}] \subseteq [M] = M$.

	Аналогичные рассуждения можно провести и для остальных классов.
\end{proof}

\subsection{Принцип двойственности}
{\bf Формулировка.}

Пусть формула $\Phi$ над $F$ задает функцию $f$. Формула $\Phi'$, получающаяся из $\Phi$ путем замены $f_i \rightarrow f_i^{*}$ будет реализовывать $f^{*}$.

$\left( \text{Напомним, что функция } f^{*}(\XXX) = \overline{f(\overline{x_1}, \ldots, \overline{x_n})} \right)$

{\bf Идея: }


$ \overline{f_0(  \overline{\overline{f_{i_1}(\ldots)}}, \ldots, \overline{\overline{ f_{i_k}(\ldots)}} )} $ -- возникаю двойные отрицания, которые изчезают, кроме {\em верхних} и {\em нижних}, что как раз дает функцию $f^{*}$.



\section{Сложность.}