\subsection{Предполные классы}

\begin{definition}
	Пусть $F \subseteq P_2$. Тогда $F$ -- предполный, если: 
	\begin{enumerate}
		\item $F \ne P_2$
		\item $[F] = F$
		\item $\forall f \notin F: [F \cup {f}] = P_2$
	\end{enumerate}
\end{definition}

В клетке таблице снизу стоит функция $\in$ строке и $\not \in $ столбцу $\Rightarrow$ ни один из классов не содержится в другом.

$$\begin{array}{|c|c|c|c|c|c|}
\hline
 & T_0 & T_1 & L & S & M  \\
\hline
T_0 &  & 0 & xy & xy & x \oplus y \\
\hline
T_1 & 1 & &  xy & xy &  x ~ y \\
\hline
L & \overline{x} & \overline{x} &  & x \oplus y & \overline{x}\\
\hline
S & \overline{x} & \overline{x} & m(x, y, z) &  & \overline{x} \\
\hline
M & 1 & 0 & xy & xy &  \\
\hline
\end{array}$$
\\
\\\\

\begin{theorem}
	$T_0, T_1, S, M, L$ -- множество всех предполных классов.
\end{theorem}

\begin{proof}
	Пусть есть $A = [A]$ -- предполный. Возьмём класс $M$.

	Пусть $A \subset M \Rightarrow \exists f \in M \setminus A : [A \cup {f}] \subseteq [M] = M$.

	Аналогичные рассуждения можно провести и для остальных классов.
\end{proof}

\subsection{Принцип двойственности}
{\bf Формулировка.}

Пусть формула $\Phi$ над $F$ задаёт функцию $f$. Формула $\Phi'$, получающаяся из $\Phi$ путём замены $f_i \rightarrow f_i^{*}$ будет реализовывать $f^{*}$.

$\left( \text{Напомним, что функция } f^{*}(\XXX) = \overline{f(\overline{x_1}, \ldots, \overline{x_n})} \right)$

{\bf Идея: }


$ \overline{f_0(  \overline{\overline{f_{i_1}(\ldots)}}, \ldots, \overline{\overline{ f_{i_k}(\ldots)}} )} $ -- возникаю двойные отрицания, которые исчезают, кроме {\em верхних} и {\em нижних}, что как раз даёт функцию $f^{*}$.


\section{Сложность.}

<<<<<<< HEAD
Последовательность элементарных операций из заданного множества операций $F$ над заданным множеством входных данных $X$.

\begin{ex}
	$X = \{x\}, F = \{x\}$

	Мы спокойно можем вычислить $x_2, x_4, x_8, x-{16}, x_{32} \ldots$

	Вычислить $x_n$ мы можем на $\leq 2\log_2(n)$, однако нам как минимум понадобится $\log_2(n)$ операций.
\end{ex}

\begin{exer}
	Для вычисления $x^n$ достаточно $\log_2(n) + O(\log_2(n))$ операций.
\end{exer}

\section{Cхемы} 

\subsection{Отступление. Графы}

Обозначим за $V = \{v_1, \ldots, v_n\}$ -- множество "вершин" нашего графа, а за $E = \{e_1, \ldots, e_k\}$ -- множество "ребер". 

$\rho : E \rightarrow V^1 \cup V^2$ (где $V^i$ -- множество всех $i-$элементных подмножеств множества $V$).

Возьмем $e \in E$. Если

\begin{itemize}
	\item $\rho(e) \in V^1$ , то $e$ -- петля.
	\item $\rho(e_1) = \rho(e_2)$ , то $e_1$ и $e_2$ -- кратные ребра.
	\item $\rho(e) = \{v_1, v_2\}$, то $v_1$ и $v_2$ -- инцидентны.
\end{itemize}

\begin{df}
	$deg(v)$ -- число ребер, инцидентных $v$ (петля дает 2).
\end{df}

\begin{itemize}
	\item Если $deg(v) = 0$, то $v$ -- изолированная вершина
	\item Если $deg(v) = 1$, то $v$ -- висячая(концевая) вершина
\end{itemize}

\begin{df}
 	\begin{itemize}
 		\item $e_1, e_2, \ldots, e_{k-1}$ -- {\bf путь}	 из $v_{i_1}$ в $v_{i_k}$, если $\rho(e_1) = \{v_{i_1}, v_{i_2}\}, \rho(e_2) = \{v_{i_2}, v_{i_2}\} , \ldots, \rho(e_{k-1}) = \{v_{i_{k-1}}, v_{i_k}\}$
 		\item Если все $v_{i_k}$ различны, то путь называется {\bf цепью}
 		\item Путь, в котором $v_{i_1} = v_{i_k}$ называется {\bf циклом}
 		\item Цикл называется простым, если в нем все вершины и ребра различны, кроме $v_{i_1} = v_{i_k}$
 	\end{itemize}
\end{df}

\begin{df}
	Граф $G = (V, E, \rho)$ называется {\bf связным}, если для любых вершин $v_1$ и $v_2$ существует путь из $v_1$ в $v_2$.

	Дерево -- связный граф без простых циклов.
\end{df}

\begin{exer}
	$G = (V, E, \rho)$ -- дерево $\Longleftrightarrow$
	\begin{enumerate}
		\item $G$ -- связен, но после удаления любого ребра -- не связен
		\item $G$ -- без простых циклов, но при добавлении любого ребра -- есть цикл
		\item Любые двее вершины соединениы единственной простой цепью
		\item $G$ -- связен, $|E| = |V| - 1$
		\item В $G$ нет простых циклов, $|E| = |V| - 1$
	\end{enumerate}
\end{exer}

\begin{df}
	Ориентированный граф -- то же самое, только $\rho: E \rightarrow V \times V$
\end{df}


\section{Схемы из функциональных элементов}

$B_0 = \{x \& y, x \vee y, \overline{x}\}$  

$X = \{\XXX\}$

СФЭ в $B_0$ -- это ориентированный граф без ориентированных циклов такой, что входящая степень каждой вершины может быть только 0, 1 или 2.

Если $deg^{+}(v) = 0$, то символ из $X$, если $deg^{+}(v) = 1$, то операция 'отрицание', если $deg^{+}(v) = 2$, то $\&$ или $\vee$.

Кроме того, одна или несколько вершин помечены $\bigstar$  -- это "выходы".

Если $deg(v) = 0$, то назовем $v$ входом.

\begin{stm}
	В любом ориентированном графе без ориентированных циклов есть $v \in V$, такая что $deg(v) = 0$
\end{stm}

\begin{stm}
	В любом ориентированном графе без ориентированных циклов с $n$ вершинами можно занумеровать вершины от 1 до $n$ так, что все ребра ведут от вершины с меньшим номером к вершине с большим номером.
\end{stm}

(*здесь должен быть пример с картинкой*)


СФЭ в базие $B$ над $X$ -- последовательность равенств

$y_1 = \varphi_1(z_{1,1}, \ldots, z_{1,r_1})$ 

$ \ldots$

$y_i = \varphi_i(z_{i,1}, \ldots, z_{i,r_i})$ 

$\ldots$

$ y_l = \varphi_l(z_{l,1}, \ldots, z_{l,r_l})$

Все $\varphi_i \in B$ и $z_{i,j} \in X \cup \{y_1, \ldots, y_{i - 1}\}$

(*Здесь тоже не хватает картинки*)

\begin{df}
	Сложность схемы -- число вершин, не являющихся входом.
\end{df}

\section{Сложность.}

	