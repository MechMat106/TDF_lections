\section{Лекция n (продолжение оценки функции Шеннона).} 
\subsection{В предыдущих сериях}
Мы уже имеем для функции Шеннона следующую оценку скорости роста:
$$n\leq L(n)\leq 6\cdot \frac{2^n}{n}$$
Попробуем теперь ее улучшить.\\
Напомним, что:
\begin{definition}
Приведенная схема -- схема, в которой все элементы выполняют разные функции.
\end{definition}
\textit{Заметим, что в данном определении под разными функциями понимаются неравные функции, либо равные функции, действующие на разном наборе переменных.}\\
Все булевы функции в этой лекции будем считать от переменных $\XXX$. Также все выкладки проводятся в выбранном "стандартном" базисе $\{\& ,\vee ,-\}$, если не указано обратное.
\subsection{Определения}
\begin{definition}
$N_{=(n,l)}$ -- число приведенных схем сложности $l$ со входами $\XXX$.
\end{definition}
\begin{definition}
$N_{\leq(n,l)}$ -- число приведенных схем сложности не выше $l$ со входами $\XXX$.
\end{definition}
\subsection{Собственно, оценка}
\begin{lemma}
При достаточно больших $n$ при $l\geq n \:\: \exists \, C>0 \: : $\\
$$N_{\leq(n,l)}\leq (C\cdot l)^l $$
\end{lemma}
Пусть $S$ -- приведенная схема сложности $l$, входы $\XXX$.
Пронумеруем элементы схемы и зафиксируем нумерацию. $L_i$ -- элемент схемы, имеющий в данной нумерации номер $i$. На данной схеме $S$ с нумерацией $Num$ введем функцию:
$(S,Num)\rightarrow T$, где $T$ -- таблица вида:
$$
    \begin{array}{|c|c|c|}
    \hline
    f & E_1 & E_2 \\
    \hline
    \& & \cdots &  \cdots \\
    \hline
     - & * & * \\
     \cdots & \cdots & \cdots \\
    \vee & e_1 & e_2 \\
     \cdots & \cdots & \cdots \\
    \hline
    \end{array}
    $$
Таблица состоит из $l$ строк и $3$ столбцов, в строчке с номером $i$ в первом столбце стоит знак функции, которую реализует элемент $L_i$, a в двух других -- то, что поступает на "вход"  этой функции. Это могут быть переменные, или "выход"  других элементов. Иначе, $e_1, \:e_2\, \in \{ L_1, \cdots ,L_l, \XXX\} $. Если функция в левом столбце -- отрицание, в два правых столбца запишем один и тот же элемент из вышеуказанного множества, над которым производится отрицание. Также отметим звездочкой строку с номером элемента, который является выходом всей схемы. \\
По такой таблице, очевидно, можно восстановить схему $S$.\\
Оценим число таблиц, соответствующих всем парам $(S,Num)$ заданной сложности $l$. Обозначим его $N_{l}$.\\
$$ N_l\leq 3^l\cdot (l+n)^{2l} \cdot l \leq 3^l \cdot 2l^{2l}\cdot l = (3\cdot 2^2)^l\cdot l^{2l}\cdot l = 12^l \cdot l^{2l} \cdot l \leq 13^l\cdot l^{2l}$$\\
\begin{statement}
Пусть схема $S$ -- приведенная, $Num_1 \neq Num_2$ -- две ее нумерации, $(S, Num_1) \rightarrow T_1$;  $(S, Num_2) \rightarrow T_2$, тогда $T_1 \neq T_2$
\end{statement}
\begin{proof}
Предположим, что $T_1 = T_2$.\\
Введем на $S$ еще одну нумерацию $Num_0$, монотонную, и зафиксируем ее. Дальше, перебирая по порядку нумерации $Num_0$ элементы схемы $S$ найдем первый элемент $L_i$ (по нумерации $Num_0$) такой, что он имеет в $Num_1$ и $Num_2$ номера $k_1$ и $k_2$, причем $k_1 \neq k_2$ Такой элемент существует, потому что $Num_1 \neq Num_2$.\\ Рассмотрим строки $k_1$ и $k_2$ таблиц $T_1$ и $T_2$ соответственно. 
В первой их клетке стоит один и тот же знак, так как функция, которую реализует элемент, не зависит от нумерации. Для двух других клеток есть две возможности: либо там стоит знак переменной, тогда они тоже одинаковы, либо элемент множества $\{ L_1 \cdots L_l\} $, для каждой таблицы в своей нумерации. \\
Посмотрим, "выход" каких элементов может подаваться на "вход" элемента $L_i$ (в нумерации $Num_0$). Так как $Num_0$ -- монотонная, то  это могут быть только элементы с меньшим номером в данной операции.  Но для элементов с меньшим номером в $Num_0$ их номера в $Num_1$ и $Num_2$ совпадают. \\
Это значит, что строчки $k_1$ в $T_1$ и $k_2$ в $T_2$ одинаковые. Так как таблицы (по нашему предположению) одинаковые, то в таблице $T_1$ строчка c номером $k_2 \neq k_1$ совпадает со строчкой с тем же номером в $T_2$, которая, в свою очередь, совпадает со строчкой $k_1$ в $T_1$. Это значит, что в $T_1$ есть две одинаковые строчки. Другими словами, в схеме $S$ есть два элемента, реализующие одинаковые функции. Это противоречит с приведенностью $S$.\\
Итак, $T_1 \neq T_2$
\end{proof}
Значит, число таблиц, соответствующей какой-либо схеме равно числу способов пронумеровать элементы этой схемы.
Тогда, учитывая, что число способов пронумеровать $l$ элементов -- $l!$, и что $l! \geq (\frac{l}{3})^l $:\\
$$ N_{=(n,l)}=\frac{N_l}{l!} \leq \frac{13^l\cdot l^{2l}}{l!} \leq 39^l\cdot l^l $$


