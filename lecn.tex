\section{Лекция n (продолжение оценки функции Шеннона).} 
\subsection{В предыдущих сериях}
Мы уже имеем для функции Шеннона следующую оценку скорости роста:
$$ L(n)\leq 6\cdot \frac{2^n}{n}$$
Попробуем теперь оценить ее снизу.\\
Напомним, что:
\begin{definition}
Приведенная схема -- схема, в которой все элементы выполняют разные функции.
\end{definition}
\textit{Заметим, что в данном определении под разными функциями понимаются неравные функции, либо равные функции, действующие на разном наборе переменных.}\\
Все булевы функции в этой лекции будем считать от переменных $\XXX$. Также все выкладки проводятся в выбранном "стандартном" базисе $\{\& ,\vee ,-\}$, если не указано обратное.
\subsection{Определения}
\begin{definition}
$N_{=(n,l)}$ -- число приведенных схем сложности $l$ со входами $\XXX$.
\end{definition}
\begin{definition}
$N_{\leq(n,l)}$ -- число приведенных схем сложности не выше $l$ со входами $\XXX$.
\end{definition}
\subsection{Собственно, оценка}
\begin{lemma}
При достаточно больших $n$ при $l\geq n \:\: \exists \, C>0 \: : $\\
$$N_{\leq(n,l)}\leq (C\cdot l)^l $$
\end{lemma}
Пусть $S$ -- приведенная схема сложности $l$, входы $\XXX$.
Пронумеруем элементы схемы и зафиксируем нумерацию. $L_i$ -- элемент схемы, имеющий в данной нумерации номер $i$. На данной схеме $S$ с нумерацией $Num$ введем функцию:
$(S,Num)\rightarrow T$, где $T$ -- таблица вида:
$$
    \begin{array}{|c|c|c|}
    \hline
    f & E_1 & E_2 \\
    \hline
    \& & \cdots &  \cdots \\
    \hline
     - & * & * \\
     \cdots & \cdots & \cdots \\
    \vee & e_1 & e_2 \\
     \cdots & \cdots & \cdots \\
    \hline
    \end{array}
    $$
Таблица состоит из $l$ строк и $3$ столбцов, в строчке с номером $i$ в первом столбце стоит знак функции, которую реализует элемент $L_i$, a в двух других -- то, что поступает на "вход"  этой функции. Это могут быть переменные, или "выход"  других элементов. Иначе, $e_1, \:e_2\, \in \{ L_1, \cdots ,L_l, \XXX\} $. Если функция в левом столбце -- отрицание, в два правых столбца запишем один и тот же элемент из вышеуказанного множества, над которым производится отрицание. Также отметим звездочкой строку с номером элемента, который является выходом всей схемы. \\
По такой таблице, очевидно, можно восстановить схему $S$.
